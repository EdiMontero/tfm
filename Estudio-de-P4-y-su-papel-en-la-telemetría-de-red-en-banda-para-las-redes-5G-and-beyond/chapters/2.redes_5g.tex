\chapter{Fundamentos de las Redes 5G y Beyond}

\section{Evolución de tecnologías móviles}
Las redes móviles han evolucionado significativamente desde su inicio, pasando por varias generaciones que han mejorado la capacidad, velocidad y latencia de las comunicaciones inalámbricas. pasando a ser una pieza clave en la infraestructura de telecomunicaciones global, impactando positivamente en diversos sectores como la industria, la salud y el entretenimiento, entre otros.
\subsection{De 1G a 4G: Hitos clave}
\item \textbf{1G}: Introducción de la comunicación analógica.
\item \textbf{2G}: Digitalización de la voz y servicios básicos de datos (SMS).
\item \textbf{3G}: Introducción de datos móviles y servicios multimedia.
\item \textbf{4G}: Redes IP y mayor velocidad de datos.
\item \textbf{LTE y LTE-Advanced}: Mejoras en eficiencia espectral y latencia.

\subsection{De 4G a 5G SA}
\item \textbf{5G Non-Standalone (NSA)}: Uso combinado de 4G y 5G para una transición suave. Aumentando la capacidad y velocidad de la red.
\item \textbf{5G Standalone (SA)}: Arquitectura completamente nueva basada en servicios y virtualización, permitiendo nuevas funcionalidades y mejoras en latencia y confiabilidad.
 
\subsection{5G Advanced y Beyond-5G}
\item \textbf{5G Advanced}: Mejoras en IA, eficiencia energética y soporte para nuevas aplicaciones, como XR y comunicaciones vehiculares.
\item \textbf{Beyond-5G}: Investigación en tecnologías emergentes como comunicaciones cuánticas y redes holográficas, con miras a la futura generación 6G.

\subsection{Tendencias hacia 6G}
\item \textbf{Visión de 6G}: Redes ultra confiables, latencia casi nula y capacidades de inteligencia artificial integradas de manera nativa.
\item \textbf{Tecnologías emergentes}: Comunicaciones terahertz, redes auto-organizadas y computación en el borde (Edge Computing).
\item \textbf{Aplicaciones futuras}: Realidad extendida (XR), ciudades inteligentes (Smart Cities) y redes de sensores masivos (IoT masivo).
\item \textbf{Desafíos}: Seguridad, privacidad y sostenibilidad ambiental.

%----------------------------------------
\section{Características de RAN para 5G y beyond}
En 5G (Release 15) el UE puede conectarse a la red mediante dos tipos de acceso: LTE (E-UTRA) y NR (New Radio). La red Core puede ser EPC (Evolved Packet Core) o 5GC (5G Core), dependiendo de si la red es NSA o SA, respectivamente. Posteriormente, esto se amplió para que ambas celdas puedan pertenecer a 5G NR, en cuyo caso la CN es exclusivamente 5G Core. Estas diversas opciones se agrupan en el término Multi-Radio Dual Connectivity (MR-DC). MR-DC es una generalizacion de Intra-E-UTRA Dual Connectivity (EN-DC) y E-UTRA-NR Dual Connectivity (NE-DC) que permite que un UE se conecte simultáneamente a dos nodos de acceso, que pueden ser E-UTRA o NR. En MR-DC, un nodo de acceso actúa como el nodo maestro o master node (MN) y el otro como el nodo secundario o secondary node (SN). El MN es responsable de la señalización y el control, mientras que el SN se utiliza principalmente para la transferencia de datos.



\section{Arquitectura 5G Non Standalone (NSA) }
La arquitectura 5G Non-Standalone (NSA) es una configuración de red que permite la coexistencia y colaboración entre las tecnologías 4G LTE y 5G NR. En esta arquitectura, la red 4G LTE actúa como la red principal para la señalización y el control, mientras que la red 5G NR se utiliza principalmente para el transporte de datos de alta velocidad. Esta configuración permite a los operadores de red aprovechar la infraestructura existente de LTE mientras implementan nuevas capacidades de NR, facilitando una transición más suave hacia las redes 5G completas.

Desde una perspectiva operacional, la arquitectura NSA permite a los proveedores de servicios móviles activar capacidades 5G de forma gradual sin necesidad de reemplazar completamente la infraestructura de núcleo de red existente. El despliegue de NSA comienza típicamente con la instalación de nuevas estaciones base NR que coexisten con las estaciones base LTE existentes. Los dispositivos del usuario (UE) que soportan tanto LTE como NR pueden conectarse simultáneamente a ambas redes, aprovechando la mejor señal disponible para la señalización de control a través de LTE y utilizando NR para el tráfico de datos cuando está disponible. Este enfoque dual proporciona beneficios inmediatos de rendimiento sin requerir una arquitectura de núcleo completamente nueva, lo que reduce significativamente los costos de inversión durante la transición.

La arquitectura NSA introduce el concepto de "Dual Connectivity", donde un dispositivo móvil mantiene conexiones simultáneas a dos estaciones base: una primaria (master) y una secundaria (secundaria). En el caso más común de NSA, la estación base LTE actúa como el nodo maestro, proporcionando la conexión primaria y controlando el plano de control, mientras que la estación base NR actúa como un nodo secundario dedicado al tráfico de datos de alta velocidad. Esta separación de funciones permite que los operadores optimicen el uso del espectro radioeléctrico, permitiendo que LTE continúe manejando eficientemente la señalización de control y la cobertura amplia, mientras que NR proporciona capacidad adicional de datos donde la densidad de usuarios es más alta.

Desde el punto de vista del usuario final, la arquitectura NSA ofrece una experiencia mejorada en comparación con las redes LTE puras. Los usuarios pueden experimentar velocidades de descarga más altas (potencialmente en el rango de gigabits por segundo), menor latencia en aplicaciones específicas de datos, y mejor eficiencia general de la red. Sin embargo, las ventajas están limitadas principalmente al tráfico de datos, ya que los servicios de señalización y control siguen estando limitados por las especificaciones de LTE. Para casos de uso que requieren baja latencia extrema o requisitos ultra confiables (como comunicaciones críticas), la arquitectura NSA puede no ser suficiente, lo que subraya la necesidad eventual de migrar a 5G SA para alcanzar todas las capacidades de 5G.

\subsection{Opciones de arquitectura 3GPP 5G NSA}
El trabajo realizado por el 3GPP sobre la arquitectura de la red 5G dio como resultado un conjunto de alternativas arquitectónicas, fundamentadas en tres decisiones clave del propio 3GPP. Este estudio esta documentado en el reporte técnico 3GPP TR 23.799 \cite{3gpp.23.799}, donde se describen las diferentes opciones arquitectónicas para la implementación de redes 5G NSA. A continuación, se presentan las principales opciones:
\begin{itemize}
    \item \textbf{Opción 1}: Arquitectura basada en LTE/EPC, donde el Core de la red sigue siendo el EPC y se utiliza NR como acceso adicional.
    \item \textbf{Opción 2}: Arquitectura basada en 5G Core (5GC), donde tanto el acceso como el Core de la red son 5G (SA).
    \item \textbf{Opción 3}: Arquitectura híbrida que combina elementos de LTE/EPC y 5GC, permitiendo una transición gradual hacia 5G.
\end{itemize}   
El hecho de que la red de acceso LTE y NR puedan coexistir en una misma red 5G NSA, permite a los operadores de red aprovechar la infraestructura existente de LTE mientras implementan nuevas capacidades de NR. Por lo que la red LTE (RAN) tiene dos formas de conectarse con el Core (Core):
\begin{itemize}
    \item \textbf{Conexión mediante S1 al EPC}: En esta configuración, la red LTE se conecta al EPC a través de la interfaz S1, mientras que la red NR se conecta al EPC a través de una interfaz adaptada. Esta opción permite una integración más sencilla con la infraestructura LTE existente.
    \item \textbf{Conexión mediante N2/N3 al Core 5GC}: En esta configuración, la red LTE se conecta al Core 5GC a través de las interfaces N2 y N3, mientras que la red NR también se conecta al 5GC. Esto permite una integración más estrecha entre las redes LTE y NR, facilitando la gestión y el control de la red.
\end{itemize}
Teniendo en cuenta que RAN de LTE y NR pueden coexistir en una misma red 5G NSA, existen cuatro formas de implementar LTE y/o NR. \cite{3gpp.23.799}:
\begin{itemize}
    \item \textbf{Opción 1}: Solo LTE para todo el trafico de datos y señalización.
    \item \textbf{Opción 2}: Solo NR para todo el trafico de datos y señalización.
    \item \textbf{Opción 3}: Una combinación de LTE y NR donde LTE tiene la mayor cobertura y se utiliza para señalización, mientras que LTE y NR se utilizan para tráfico de datos.
    \item \textbf{Opción 4}: Una combinación de LTE y NR donde NR tiene la mayor cobertura y se utiliza para señalización, mientras que LTE y NR se utilizan para tráfico de datos.
\end{itemize}
Considerando la integración de 2 Core diferentes (EPC y 5GC) en una red 5G NSA, existen 8 formas de implementar la arquitectura 5G NSA. \cite{3gpp.sp.160455}.

%-------Imagen Figura 2.1-------
\begin{figure}[h]
    \centering
    \includegraphics[width=1.0\textwidth]{images/combinacion-ran.png}
    \caption{Posibles combinaciones de RAN y Core en una red 5G NSA \cite{3gpp.sp.160455}.}
    \label{fig:5g-combinations}
\end{figure}
%-------------------------------
En la figura \ref{fig:5g-combinations} se muestran las diferentes combinaciones posibles de RAN y Core en una red 5G NSA 4 × 2 = 8. Las cuales son:
\begin{itemize}
    \item \textbf{Opción 1}: RAN LTE conectada al EPC. (Solo LTE)
    \item \textbf{Opción 2}: RAN LTE conectada al 5GC. (Solo LTE)
    \item \textbf{Opción 3}: RAN LTE y NR conectadas al EPC. (LTE para señalización y datos, NR para datos)
    \item \textbf{Opción 4}: RAN LTE y NR conectadas al 5GC. (LTE para señalización y datos, NR para datos)
    \item \textbf{Opción 5}: RAN NR y LTE conectadas al EPC. (NR para señalización y datos, LTE para datos)
    \item \textbf{Opción 6}: RAN NR y LTE conectadas al 5GC. (NR para señalización y datos, LTE para datos)
    \item \textbf{Opción 7}: RAN NR conectada al EPC. (Solo NR)
    \item \textbf{Opción 8}: RAN NR conectada al 5GC. (Solo NR)
\end{itemize}
Las opciones 3, 4, 5 y 6 son las que permiten la coexistencia de LTE y NR en la misma red 5G NSA, aprovechando las capacidades de ambas tecnologías para mejorar la cobertura y el rendimiento de la red. Estas opciones representan diferentes estrategias de integración que los operadores pueden adoptar según sus objetivos de despliegue y requisitos de infraestructura.

En las opciones 3 y 4, LTE actúa como la tecnología principal para la señalización de control y la cobertura amplia, mientras que NR se despliega como una tecnología complementaria para proporcionar capacidad adicional de datos de alta velocidad en áreas densamente pobladas. Esta configuración permite a los operadores mantener una cobertura LTE extendida (que puede alcanzar kilómetros de distancia) mientras aprovechan las bandas de frecuencia más altas de NR (donde la propagación es más limitada) para aumentar la capacidad de datos. Las opciones 3 y 4 difieren en el núcleo de red utilizado: la opción 3 utiliza el EPC tradicional de 4G, lo que requiere menos cambios en la infraestructura existente, mientras que la opción 4 utiliza el 5GC, permitiendo una integración más profunda con las funciones de red 5G.

Las opciones 5 y 6 invierten el rol de las tecnologías, donde NR se convierte en la tecnología primaria para la señalización y el control, mientras que LTE actúa como una tecnología secundaria para datos adicionales. Este enfoque es útil en escenarios donde los operadores han desplegado ampliamente NR y desean optimizar la utilización del espectro LTE existente o en regiones donde LTE proporciona una cobertura superior. Similar a las opciones anteriores, la opción 5 utiliza EPC como núcleo de red, mientras que la opción 6 utiliza 5GC, proporcionando mayor flexibilidad en la arquitectura de red.

La selección entre estas opciones (3, 4, 5 o 6) depende de varios factores estratégicos: la cobertura existente de LTE en la región, la disponibilidad de espectro NR, la madurez de la infraestructura 5GC disponible, los objetivos de calidad de servicio requeridos para diferentes tipos de aplicaciones, y las inversiones ya realizadas en infraestructura de red anterior. Los operadores típicamente comienzan con las opciones 3 o 4 (donde LTE es primario) para aprovechar su cobertura establecida, y gradualmente evolucionan hacia las opciones 5 o 6 (donde NR es primario) conforme la cobertura de NR se expande y se vuelve más dominante en la red.
%-----------------------------------------












\section{Arquitectura 5G Standalone (SA)}
\subsection{Visión general de la arquitectura 5G SA}
La arquitectura 5G Standalone (SA) representa una evolución significativa en comparación con las generaciones anteriores de redes móviles. A diferencia de las implementaciones Non-Standalone (NSA), que dependen de la infraestructura existente de 4G LTE, la arquitectura SA está diseñada desde cero para aprovechar al máximo las capacidades de la tecnología 5G. Esta arquitectura se basa en una serie de principios clave que permiten una mayor flexibilidad, eficiencia y capacidad para soportar una amplia gama de servicios y aplicaciones. Entre los aspectos más destacados de la arquitectura 5G SA se encuentran:
\begin{itemize}
    \item \textbf{Red basada en servicios}: La arquitectura 5G SA adopta un enfoque basado en servicios, donde las funciones de red se implementan como servicios independientes que pueden ser orquestados y gestionados de manera flexible.
    \item \textbf{Virtualización y desagregación}: La arquitectura permite la virtualización de funciones de red (NFV) y la desagregación de hardware y software, lo que facilita la implementación en entornos de nube y mejora la escalabilidad.
    \item \textbf{Separación del plano de control y plano de usuario}: Esta separación permite una gestión más eficiente del tráfico y una mejor calidad de servicio (QoS) para diferentes tipos de aplicaciones.
    \item \textbf{Soporte para nuevas tecnologías}: La arquitectura 5G SA está diseñada para integrar tecnologías emergentes como la inteligencia artificial (IA), el edge computing y la Internet de las cosas (IoT).
\end{itemize}

\subsection{Modos de implementación de 5G SA}
Existen tres modos principales de implementación de redes 5G NSA, que permiten la coexistencia de tecnologías LTE y NR en una misma red. Estos modos son:
\begin{itemize}
    \item \textbf{NGEN-DC (NG-RAN E-UTRA-NR Dual Connectivity)}: el UE (User Equipment) se conecta simultáneamente a una estación base LTE (eNodeB) como master node (MN) y a una estación base NR (gNodeB) como secondary node (SN), utilizando el EPC como Core de la red. El gNodeB (en-gNB) se puede conectar al EPC mediante la interface S1-U o x2-U para el plano de usuario y la interfaz S1-MME o X2-C para el plano de control.
    \item \textbf{NE-DC (NR-E-UTRA Dual Connectivity)}: el UE se conecta simultáneamente a una estación base NR como master node (MN) y a una estación base LTE como secondary node (SN), utilizando el 5GC como Core de la red.
    \item \textbf{NR-DC (NR-NR Dual Connectivity)}: el UE se conecta simultáneamente a dos estaciones base NR, una como master node (MN) y la otra como secondary node (SN), utilizando el 5GC como Core de la red.
\end{itemize}
La descripcion de DC se encuentra en el documento 3GPP TS 37.340 \cite{3gpp.ts.37.340}.




\subsection{Perspectivas en 5GC}
La arquitectura del núcleo de red 5G (5GC) introduce una serie de innovaciones y mejoras en comparación con las arquitecturas de núcleo de red anteriores, como el EPC utilizado en 4G. Estas mejoras están diseñadas para soportar las demandas crecientes de conectividad, velocidad y baja latencia que caracterizan a las redes 5G.

\section{Arquitectura basada en servicios (SBA)}
La diferencia más destacada en 5G frente a las arquitecturas 3GPP anteriores es la adopción del concepto de interfaces basadas en servicios \textbf{(SBA)} . Esto implica que las funciones de red que contienen la lógica y los mecanismos para procesar los flujos de señalización ya no se conectan mediante interfaces punto a punto, sino que \textbf{exponen sus capacidades como servicios} accesibles para otras funciones de red. En cada intercambio, una función actúa como \textbf{consumidora de servicios} y la otra como \textbf{proveedora de dichos servicios}. \ref{fig:5g-sba} muestra la arquitectura 5GC basada en SBA.

%-------Imagen Figura 2.2-------
\begin{figure}[H]
    \centering
    \includegraphics[width=1.0\textwidth]{images/sba.png}
    \caption{Arquitectura 5GC basada en interfaces basadas en servicios. \cite{oreillyf2021}.}
    \label{fig:5g-sba}
\end{figure}
%-------------------------------

Las funciones de red en 5GC se comunican entre sí a través de una interfaz común llamada \textbf{Service-Based Interface (SBI)}. Esta interfaz utiliza protocolos estándar como HTTP/2 y RESTful APIs para facilitar la comunicación entre las funciones de red. La adopción de SBA permite una mayor flexibilidad y escalabilidad en la arquitectura de la red, ya que las funciones pueden ser desarrolladas, desplegadas y actualizadas de manera independiente. Además, esta arquitectura facilita la integración con tecnologías emergentes como la virtualización de funciones de red (NFV) y la computación en la nube, permitiendo a los operadores de red adaptarse rápidamente a las demandas cambiantes del mercado y ofrecer nuevos servicios de manera más eficiente.

La arquitectura SBA también se puede representar como punto a punto, donde cada función de red se conecta directamente con las demás funciones que requieren sus servicios, utilizando la interfaz SBI para la comunicación. Como se muestra en la figura \ref{fig:sbi_p2p}.
%-------Imagen Figura 2.3-------
\begin{figure}[H]
    \centering
    \includegraphics[width=1.0\textwidth]{images/sbi_p2p.png}
    \caption{Arquitectura 5GC con interfaces punto a punto. \cite{oreillyf2021}.}
    \label{fig:sbi_p2p}
\end{figure}
%-------------------------------

\subsection{Interfaces HTTP REST}
En la arquitectura 5G Core (5GC), las funciones de red se comunican entre sí utilizando una interfaz basada en servicios conocida como Service-Based Interface (SBI). Esta interfaz utiliza el protocolo HTTP/2 junto con RESTful APIs para facilitar la comunicación y el intercambio de información entre las diferentes funciones de red. A continuación, se describen los aspectos clave de las interfaces HTTP REST en 5GC:
\begin{itemize}
    \item \textbf{Protocolo HTTP/2}: 5GC utiliza HTTP/2 como el protocolo de transporte para las comunicaciones entre funciones de red. HTTP/2 ofrece varias ventajas sobre su predecesor, HTTP/1.1, incluyendo una mayor eficiencia en la multiplexación de solicitudes, compresión de encabezados y reducción de la latencia, lo que es crucial para las aplicaciones de baja latencia en 5G.
    \item \textbf{RESTful APIs}: Las funciones de red en 5GC exponen sus capacidades a través de RESTful APIs, que son interfaces basadas en principios REST (Representational State Transfer). Estas APIs permiten a las funciones de red interactuar de manera sencilla y estandarizada, utilizando métodos HTTP como GET, POST, PUT y DELETE para realizar operaciones sobre los recursos.
    \item \textbf{Formato de datos JSON}: La comunicación entre funciones de red a través de las RESTful APIs generalmente utiliza JSON (JavaScript Object Notation) como formato de datos para el intercambio de información. JSON es ligero y fácil de leer, lo que facilita la integración entre diferentes sistemas y tecnologías.
    \item \textbf{Seguridad}: La seguridad es un aspecto crítico en las comunicaciones entre funciones de red. En 5GC, se implementan mecanismos de autenticación y autorización para garantizar que solo las funciones autorizadas puedan acceder a los servicios expuestos a través de las RESTful APIs. Además, se utilizan protocolos seguros como TLS (Transport Layer Security) para proteger la integridad y confidencialidad de los datos transmitidos.
    \item \textbf{Escalabilidad y flexibilidad}: La adopción de interfaces HTTP REST permite una mayor escalabilidad y flexibilidad en la arquitectura 5GC. Las funciones de red pueden ser desarrolladas, desplegadas y actualizadas de manera independiente, lo que facilita la adaptación a las demandas cambiantes del mercado y la incorporación de nuevas tecnologías.
\end{itemize}

\section{Componentes de 5G Core (5GC)}
En 5G Core (5GC), los componentes tienes funciones específicas a diferencia de las arquitecturas anteriores, las cuales combinaban algunas funciones como por ejemplo, el MME y el SGW en una sola entidad llamada AMF.

\subsection{Componentes principales de 5GC}
\subsubsection{NRF}
El Network Repository Function (NRF) es un componente clave en la arquitectura 5G Core (5GC) que actúa como un repositorio centralizado para la gestión y descubrimiento de servicios de red. Su función principal es mantener un registro actualizado de todas las funciones de red disponibles en la red 5G, permitiendo que otras funciones de red puedan descubrir y comunicarse con ellas de manera eficiente. acorde a \cite{3gpp.ts.23.501}.
Algunas de las funciones principales del NRF incluyen:
\begin{itemize}
    \item \textbf{Registro de instancias de NF}: El NRF mantiene un catálogo dinamizado de instancias de funciones de red (NF) con sus perfiles, incluyendo identidad (NF Instance ID), tipo de NF, direcciones de servicio (Service URIs) y versiones de API soportadas. Soporta mecanismos de heartbeat y actualización periódica del estado operacional.
    \item \textbf{Service Discovery y Load Balancing}: Las funciones de red consultan al NRF mediante APIs REST (GET /nf-instances?nf-type=SMF) para descubrir instancias disponibles. El NRF implementa algoritmos de balanceo de carga y selección basados en capacidad, latencia y carga actual de las NF.
    \item \textbf{Gestión del ciclo de vida de NF}: Controla el registro (PUT), actualización (PATCH) y desregistro (DELETE) de instancias de NF. Mantiene información de estado (REGISTERED, SUSPENDED, UNDISCOVERABLE) y maneja timeouts de expiración de registros.
    \item \textbf{Autenticación y autorización OAuth2}: Implementa OAuth2/OpenID Connect para garantizar que solo NF autorizadas accedan al catálogo de servicios. Emite y valida tokens JWT (JSON Web Tokens) con claims de autorización específicos.
    \item \textbf{Observabilidad y monitoreo}: El NRF expone métricas sobre disponibilidad de servicios, frecuencia de consultas de descubrimiento y latencia de respuesta. Soporta notificaciones (subscriptions) sobre cambios en la disponibilidad de servicios mediante mecanismo webhook.
\end{itemize}

\subsubsection{AMF}
El Access and Mobility Management Function (AMF) es una de las funciones clave en la arquitectura 5G Core (5GC) y se encarga de gestionar el acceso y la movilidad de los dispositivos de usuario (UE) en la red 5G. El AMF desempeña un papel fundamental en la gestión de la conexión del UE, la autenticación, la autorización y el control de movilidad, asegurando que los dispositivos puedan acceder a los servicios de red de manera eficiente y segura. De acuerdo a \cite{3gpp.ts.23.501}, Algunas de las funciones principales del AMF incluyen:
\begin{itemize}
    \item \textbf{Gestión de conexión del UE}: Maneja los procedimientos de registro (Registration), autenticación (Authentication) y autorización (Authorization) mediante interacción con AUSF y UDM. Establece y mantiene el contexto de conexión del UE (AMF-UE-NGAP-ID) y gestiona la liberación orderly de conexiones mediante procedimientos de deregistration.
    \item \textbf{Control de movilidad}: Implementa procedimientos de Mobility Management como handover intra-AMF, inter-AMF y entre tecnologías de acceso (NR a LTE y viceversa). Gestiona la reubicación del UE-NGAP context cuando cruza límites de cobertura, coordinando con RAN para minimizar downtime mediante procedimientos de Xn handover (inter-gNB) y N2 handover.
    \item \textbf{Interacción con funciones de red}: Coordina con SMF para establecer/modificar/liberar Session Management Subscription (SMS) a través de la interface Namf. Comunica con UPF indirectamente vía SMF para la actualización de rutas del plano de usuario (Uplink Classification Rules, Traffic Steering Rules).
    \item \textbf{Gestión de seguridad}: Implementa mecanismos de NAS (Non-Access Stratum) seguridad incluyendo NAS-MAC, NAS-ENC y derivación de claves de AS (RRC/PDCP) a partir de claves de NAS. Participa en procedimientos 5G-AKA con AUSF para mutua autenticación y derivación de claves (K\_AUSF, K\_AMF).
    \item \textbf{Soporte multi-RAT}: Gestiona contextos de conectividad dual (EN-DC, NE-DC, NR-DC) manteniendo sincronización de estado entre nodos maestro y secundario. Controla transiciones RAN-based en NSA mediante coordinación con eNodeB y gNodeB para cambios de plano de usuario sin interrupción.
\end{itemize}

\subsubsection{SMF}
El Session Management Function (SMF) se encarga de gestionar las sesiones de datos del usuario (UE) en la red 5G. El SMF desempeña un papel fundamental en la configuración, mantenimiento y liberación de las sesiones de datos, asegurando que los dispositivos puedan acceder a los servicios de red de manera eficiente y segura. De acuerdo a \cite{3gpp.ts.23.501}, algunas de las funciones principales del SMF incluyen:
\begin{itemize}
    \item \textbf{Gestión de sesiones de datos}: Es responsable de establecer, modificar y liberar sesiones de datos del UE. Esto incluye la asignación de recursos de red, la configuración de rutas del plano de usuario (UPF) y la aplicación de políticas de calidad de servicio (QoS) para garantizar un rendimiento óptimo.
    \item \textbf{Interacción con funciones de red}: Interactúa con otras funciones de red, como el AMF para la gestión del contexto del UE y el UPF para la configuración del plano de usuario. También se comunica con el PCF para aplicar políticas de QoS y control de tráfico.
    \item \textbf{Gestión de políticas de QoS}: Aplica políticas de calidad de servicio (QoS) para las sesiones de datos del UE, asegurando que se cumplan los requisitos de rendimiento y latencia para diferentes tipos de tráfico. Esto incluye la asignación de parámetros QoS como 5QI (5G QoS Identifier) y QFI (QoS Flow Identifier).
    \item \textbf{Soporte para múltiples tecnologías de acceso}: Es capaz de gestionar sesiones de datos para dispositivos que utilizan diferentes tecnologías de acceso, como NR (New Radio) y LTE (Long Term Evolution). Esto permite una mayor flexibilidad en la gestión de recursos y la optimización del rendimiento de la red.
    \item \textbf{Gestión de políticas de seguridad}: Implementa políticas de seguridad para las sesiones de datos del UE, incluyendo la autenticación y autorización de dispositivos, así como la protección de datos mediante cifrado y mecanismos de integridad. Esto garantiza que las comunicaciones sean seguras y cumplan con los estándares de privacidad.

\subsubsection{UPF}
El User Plane Function (UPF) es responsable de la gestión del plano de usuario (User Plane) en la arquitectura 5GC, actuando como el punto de anclaje para el procesamiento y enrutamiento de tráfico de datos. El UPF desempeña un papel crítico en la cadena de procesamiento de paquetes, implementando funciones de forwarding, encapsulación y aplicación de políticas a nivel de flujo. De acuerdo a \cite{3gpp.ts.23.501}, las funciones técnicas principales del UPF incluyen:
\begin{itemize}
    \item \textbf{Procesamiento del plano de usuario}: Implementa funciones de IP packet routing and forwarding, encapsulación GTP-U para tráfico entre gNodeB y UPF, así como procesamiento de headers IP y detección de flujos según Uplink Classification Rules (UCR) y Traffic Steering Rules (TSR). Gestiona la terminación de túneles GTP-U en la interfaz N3 (gNodeB-UPF) y N9 (UPF-UPF).
    \item \textbf{Aplicación de políticas de QoS}: Implementa mecanismos de Traffic Scheduling, Policing y Shaping en función de los parámetros QoS definidos (5QI, QFI, GFBR, MFBR). Realiza Downlink Packet Buffering y reordenamiento de paquetes en casos de cambio de ruta (UPF relocation). Soporta múltiples QoS Flows por PDU Session, permitiendo diferenciación granular de tráfico.
    \item \textbf{Interacción con funciones de control}: Se comunica con el SMF a través de la interfaz N4 utilizando protocolo PFCP (Packet Forwarding Control Protocol) para recibir reglas de forwarding, detección de flujos y políticas de QoS. Mantiene sesiones PFCP para cada PDU Session activa, con soporte para modificaciones dinámicas mediante PFCP Association Heartbeat.
    \item \textbf{Funciones avanzadas de forwarding}: Implementa Source IP Address Spoofing Detection, UL/DL Traffic Forwarding según sesión PFCP, Packet Duplication para garantizar entrega confiable, y Redundant Transmission en configuraciones de múltiples UPF. Soporta Traffic Steering mediante N6 routing y Load Balancing entre múltiples instancias de UPF.
    \item \textbf{Seguridad y monitoreo}: Implementa filtrado de paquetes (Access Control Lists), detección de anomalías de tráfico y generación de eventos de cambio de estado para notificación a SMF. Soporta IPDR (IP Detail Records) para facturación y análisis de uso, exposición de métricas de plano de usuario para observabilidad (packet loss, latency, throughput).
\end{itemize}
\subsubsection{AUSF}
El Authentication Server Function (AUSF) Se encarga de gestionar la autenticación de los dispositivos de usuario (UE) en la red 5G. El AUSF desempeña un papel fundamental en la seguridad de la red, asegurando que solo los dispositivos autorizados puedan acceder a los servicios de red. De acuerdo a \cite{3gpp.ts.23.501}, algunas de las funciones principales del AUSF incluyen:
\begin{itemize}
    \item \textbf{Autenticación de dispositivos}: Responsable de autenticar los dispositivos de usuario (UE) que intentan acceder a la red 5G. Esto incluye la verificación de las credenciales del dispositivo y la validación de su identidad mediante mecanismos de autenticación seguros.
    \item \textbf{Interacción con funciones de red}: Interactúa con otras funciones de red, como el AMF para la gestión del contexto del UE y el UDM para la obtención de datos de suscripción y autenticación. También se comunica con el PCF para aplicar políticas de seguridad y control de acceso.
    \item \textbf{Soporte para múltiples métodos de autenticación}: Es capaz de gestionar diferentes métodos de autenticación, incluyendo 5G-AKA (Authentication and Key Agreement), EAP (Extensible Authentication Protocol) y otros mecanismos basados en estándares. Esto permite una mayor flexibilidad en la autenticación de dispositivos y usuarios.
    \item \textbf{Generación y distribución de claves}: Genera y distribuye claves criptográficas utilizadas para proteger las comunicaciones entre el UE y la red. Estas claves son esenciales para garantizar la confidencialidad e integridad de los datos transmitidos en la red 5G.
    \item \textbf{Gestión de políticas de seguridad}: Implementa políticas de seguridad para la autenticación de dispositivos, incluyendo la gestión de credenciales, la protección contra ataques de suplantación y la prevención de accesos no autorizados. Esto garantiza que las comunicaciones sean seguras y cumplan con los estándares de privacidad.
\end{itemize}
\subsubsection{UDM}
El Unified Data Management (UDM) Es responsable de gestionar los datos de suscripción y la información del usuario en la red 5G. El UDM desempeña un papel fundamental en la gestión de la identidad del usuario, las políticas de acceso y las configuraciones de servicio, asegurando que los dispositivos puedan acceder a los servicios de red de manera eficiente y segura. De acuerdo a \cite{3gpp.ts.23.501}, algunas de las funciones principales del UDM incluyen:
\begin{itemize}
    \item \textbf{Gestión de datos de suscripción}: Responsable de almacenar y gestionar los datos de suscripción del usuario, incluyendo la información de identidad, las políticas de acceso y las configuraciones de servicio. Esto permite una gestión centralizada y eficiente de los datos del usuario en la red 5G.
    \item \textbf{Interacción con funciones de red}: Interactúa con otras funciones de red, como el AMF para la gestión del contexto del UE y el AUSF para la autenticación de dispositivos. También se comunica con el PCF para aplicar políticas de acceso y control de tráfico.
    \item \textbf{Soporte para múltiples tecnologías de acceso}: Es capaz de gestionar datos de suscripción para dispositivos que utilizan diferentes tecnologías de acceso, como NR (New Radio) y LTE (Long Term Evolution). Esto permite una mayor flexibilidad en la gestión de recursos y la optimización del rendimiento de la red.
    \item \textbf{Generación y distribución de claves}: Genera y distribuye claves criptográficas utilizadas para proteger las comunicaciones entre el UE y la red. Estas claves son esenciales para garantizar la confidencialidad e integridad de los datos transmitidos en la red 5G.
    \item \textbf{Gestión de políticas de acceso}: Implementa políticas de acceso para los dispositivos de usuario, incluyendo la autorización de acceso a servicios y la aplicación de restricciones basadas en las condiciones de la red y las necesidades del usuario. Esto garantiza que los dispositivos puedan acceder a los servicios de red de manera segura y eficiente.
\end{itemize}
\subsubsection{PCF}
El Policy Control Function (PCF) Gestiona las políticas de control y calidad de servicio (QoS) en la red 5G. El PCF desempeña un papel fundamental en la aplicación de políticas de acceso, control de tráfico y gestión de recursos, asegurando que los dispositivos puedan acceder a los servicios de red de manera eficiente y segura. De acuerdo a \cite{3gpp.ts.23.501}, algunas de las funciones principales del PCF incluyen:
\begin{itemize}
    \item \textbf{Gestión de políticas de QoS}: Responsable de definir y aplicar políticas de calidad de servicio (QoS) para los dispositivos de usuario (UE) en la red 5G. Esto incluye la asignación de recursos de red, la configuración de parámetros QoS como 5QI (5G QoS Identifier) y QFI (QoS Flow Identifier), y la priorización del tráfico según las necesidades del usuario.
    \item \textbf{Interacción con funciones de red}: Interactúa con otras funciones de red, como el AMF para la gestión del contexto del UE y el SMF para la configuración del plano de usuario. También se comunica con el UDM para obtener datos de suscripción y aplicar políticas de acceso.
    \item \textbf{Soporte para múltiples tecnologías de acceso}: Es capaz de gestionar políticas de QoS para dispositivos que utilizan diferentes tecnologías de acceso, como NR (New Radio) y LTE (Long Term Evolution). Esto permite una mayor flexibilidad en la gestión de recursos y la optimización del rendimiento de la red.
    \item \textbf{Aplicación de políticas de control de tráfico}: Implementa políticas de control de tráfico para gestionar el uso de recursos de red y garantizar un rendimiento óptimo. Esto incluye la monitorización del tráfico, la aplicación de restricciones basadas en las condiciones de la red y la priorización del tráfico según las necesidades del usuario.
    \item \textbf{Gestión de seguridad}: Implementa políticas de seguridad para el control de acceso y la protección de datos en la red 5G. Esto incluye la autenticación y autorización de dispositivos, así como la protección de datos mediante cifrado y mecanismos de integridad. Esto garantiza que las comunicaciones sean seguras y cumplan con los estándares de privacidad.
\end{itemize}
\subsubsection{NSSF}
El Network Slice Selection Function (NSSF) Se encarga de gestionar la selección y asignación de redes de corte (slices) para los dispositivos de usuario (UE) en la red 5G. El NSSF desempeña un papel fundamental en la implementación de redes de corte, que permiten a los operadores ofrecer servicios personalizados y optimizados para diferentes tipos de aplicaciones y usuarios. De acuerdo a \cite{3gpp.ts.23.501}, algunas de las funciones principales del NSSF incluyen:
\begin{itemize}
    \item \textbf{Selección de redes de corte}: Responsable de seleccionar la red de corte adecuada para un dispositivo de usuario (UE) en función de sus requisitos de servicio, políticas de QoS y condiciones de la red. Esto permite a los operadores ofrecer servicios personalizados y optimizados para diferentes tipos de aplicaciones y usuarios.
    \item \textbf{Interacción con funciones de red}: Interactúa con otras funciones de red, como el AMF para la gestión del contexto del UE y el SMF para la configuración del plano de usuario. También se comunica con el PCF para aplicar políticas de QoS y control de tráfico en las redes de corte seleccionadas.
    \item \textbf{Soporte para múltiples tecnologías de acceso}: Es capaz de gestionar redes de corte para dispositivos que utilizan diferentes tecnologías de acceso, como NR (New Radio) y LTE (Long Term Evolution). Esto permite una mayor flexibilidad en la gestión de recursos y la optimización del rendimiento de la red.
    \item \textbf{Gestión de políticas de QoS en redes de corte}: Implementa políticas de calidad de servicio (QoS) específicas para cada red de corte, asegurando que se cumplan los requisitos de rendimiento y latencia para diferentes tipos de tráfico. Esto incluye la asignación de parámetros QoS como 5QI (5G QoS Identifier) y QFI (QoS Flow Identifier) en función de las necesidades del usuario y las condiciones de la red.
    \item \textbf{Gestión de seguridad en redes de corte}: Implementa políticas de seguridad para las redes de corte, incluyendo la autenticación y autorización de dispositivos, así como la protección de datos mediante cifrado y mecanismos de integridad. Esto garantiza que las comunicaciones en las redes de corte sean seguras y cumplan con los estándares de privacidad.
\end{itemize}
\subsubsection{NEF}
El Network Exposure Function (NEF) Expone las capacidades y servicios de la red 5G a aplicaciones externas y terceros. El NEF desempeña un papel fundamental en la facilitación de la interoperabilidad entre la red 5G y aplicaciones de terceros, permitiendo a los desarrolladores crear aplicaciones innovadoras que aprovechen las capacidades avanzadas de la red 5G. De acuerdo a \cite{3gpp.ts.23.501}, algunas de las funciones principales dincluyen:
\begin{itemize}
    \item \textbf{Exposición de servicios de red}: Responsable de exponer las capacidades y servicios de la red 5G a aplicaciones externas, permitiendo a los desarrolladores acceder a funciones como la gestión de sesiones, la calidad de servicio (QoS) y la movilidad del usuario. Esto facilita la creación de aplicaciones innovadoras que aprovechen las capacidades avanzadas de la red 5G.
    \item \textbf{Interacción con funciones de red}: Interactúa con otras funciones de red, como el AMF para la gestión del contexto del UE y el SMF para la configuración del plano de usuario. También se comunica con el PCF para aplicar políticas de QoS y control de tráfico en las aplicaciones expuestas.
    \item \textbf{Soporte para múltiples tecnologías de acceso}: Capaz de exponer servicios de red para dispositivos que utilizan diferentes tecnologías de acceso, como NR (New Radio) y LTE (Long Term Evolution). Esto permite una mayor flexibilidad en la creación de aplicaciones y la optimización del rendimiento de la red.
    \item \textbf{Gestión de políticas de QoS en aplicaciones expuestas}: Implementa políticas de calidad de servicio (QoS) específicas para las aplicaciones expuestas, asegurando que se cumplan los requisitos de rendimiento y latencia para diferentes tipos de tráfico. Esto incluye la asignación de parámetros QoS como 5QI (5G QoS Identifier) y QFI (QoS Flow Identifier) en función de las necesidades del usuario y las condiciones de la red.
    \item \textbf{Gestión de seguridad en aplicaciones expuestas}: Implementa políticas de seguridad para las aplicaciones expuestas, incluyendo la autenticación y autorización de dispositivos, así como la protección de datos mediante cifrado y mecanismos de integridad. Esto garantiza que las comunicaciones entre las aplicaciones externas y la red 5G sean seguras y cumplan con los estándares de privacidad.
\end{itemize}
\subsubsection{AF}
El Application Function (AF) Gestiona las aplicaciones que interactúan con la red 5G. El AF desempeña un papel fundamental en la facilitación de la comunicación entre las aplicaciones y las funciones de red, permitiendo a los desarrolladores crear aplicaciones innovadoras que aprovechen las capacidades avanzadas de la red 5G. De acuerdo a \cite{3gpp.ts.23.501}, algunas de las funciones principales del AF incluyen:
\begin{itemize}
    \item \textbf{Gestión de aplicaciones}: Maneja las aplicaciones que interactúan con la red 5G, incluyendo la configuración, el monitoreo y la optimización del rendimiento de las aplicaciones. Esto permite a los desarrolladores crear aplicaciones innovadoras que aprovechen las capacidades avanzadas de la red 5G.
    \item \textbf{Interacción con funciones de red}: Interactúa con otras funciones de red, como el AMF para la gestión del contexto del UE y el SMF para la configuración del plano de usuario. También se comunica con el PCF para aplicar políticas de QoS y control de tráfico en las aplicaciones gestionadas.
    \item \textbf{Soporte para múltiples tecnologías de acceso}: Es capaz de gestionar aplicaciones para dispositivos que utilizan diferentes tecnologías de acceso, como NR (New Radio) y LTE (Long Term Evolution). Esto permite una mayor flexibilidad en la creación de aplicaciones y la optimización del rendimiento de la red.
    \item \textbf{Gestión de políticas de QoS en aplicaciones}: Implementa políticas de calidad de servicio (QoS) específicas para las aplicaciones gestionadas, asegurando que se cumplan los requisitos de rendimiento y latencia para diferentes tipos de tráfico. Esto incluye la asignación de parámetros QoS como 5QI (5G QoS Identifier) y QFI (QoS Flow Identifier) en función de las necesidades del usuario y las condiciones de la red.
    \item \textbf{Gestión de seguridad en aplicaciones}: Implementa políticas de seguridad para las aplicaciones gestionadas, incluyendo la autenticación y autorización de dispositivos, así como la protección de datos mediante cifrado y mecanismos de integridad. Esto garantiza que las comunicaciones entre las aplicaciones y la red 5G sean seguras y cumplan con los estándares de privacidad.
\end{itemize}




\subsection{Funciones del plano de control y plano de usuario}
Plano de Control (Control Plane - CP): El plano de control es responsable de la gestión y control de las conexiones de red, incluyendo la señalización, la autenticación, la autorización y la gestión de movilidad. Las funciones del plano de control incluyen:
\begin{itemize}
    \item Gestión de la conexión del usuario (UE)
    \item Autenticación y autorización
    \item Gestión de movilidad y handovers
    \item Configuración y gestión de sesiones
    \item Aplicación de políticas de calidad de servicio (QoS)
\end{itemize}
\subsubsection{Pila de protocolos 5GC}
A continuacion se muestra la pila de protocolos 
%-------Imagen Figura 2.3-------
%\begin{figure}[H]
%    \centering
%    \includegraphics[width=1.0\textwidth]{images/sbi_p2p.png}
%    \caption{Arquitectura 5GC con interfaces punto a punto. \cite{oreillyf2021}.}
%    \label{fig:sbi_p2p}
%\end{figure}
%-------------------------------


Plano de Usuario (User Plane - UP): El plano de usuario es responsable del transporte de datos entre el usuario y la red. Las funciones del plano de usuario incluyen:
\begin{itemize}
    \item Enrutamiento y reenvío de paquetes de datos
    \item Gestión del tráfico y aplicación de políticas de QoS
    \item Procesamiento de datos, como la inspección profunda de paquetes (DPI)
    \item Soporte para funciones avanzadas, como la duplicación de paquetes y la transmisión redundante
\end{itemize}

\subsubsection{Pila de protocolos 5GC}
A continuación se muestra la pila de protocolos del plano de usuario de forma detallada \ref{fig:stack-up}.

%-------Imagen Figura 2.3-------
\begin{figure}[H]
    \centering
    \includegraphics[width=1.0\textwidth]{images/protocol-stack-up.png}
    \caption{Pila de protocolos del plano de usuario en 5GC. \cite{3gpp.23.501, 3gpp.ts.38.300}.}
    \label{fig:stack-up}
\end{figure}
%-------------------------------




\subsection{Interfaces clave: N1, N2, N3, N4, N6, N9}
\subsection{NGAP y N2}
\subsection{NGRAN y separación CP/UP}

\section{Protocolos relevantes en 5G}
\subsection{NGAP}
\subsection{SCTP en 5G}
\subsection{GTP-U}
\subsection{PFCP}

\section{QoS en 5G}
\subsection{Conceptos de QFI, 5QI y políticas de flujo}
\subsection{Programación del tráfico en UPF}
\subsection{QoS-based routing y desafíos actuales}
\subsection{Relación entre QoS y telemetría}
