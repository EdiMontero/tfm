\documentclass[12pt,a4paper]{report}

%-------------------------------
% Paquetes básicos
%-------------------------------
\usepackage[spanish]{babel}
\usepackage[utf8]{inputenc}
\usepackage[T1]{fontenc}
\usepackage{graphicx}
\usepackage{hyperref}
\usepackage{amsmath, amssymb}
\usepackage{listings}
\usepackage{transparent}
\usepackage{eso-pic}
\usepackage{booktabs}
\usepackage{float}

%-------------------------------
% Encabezados y pies (logo UPCT)
%-------------------------------
\usepackage{fancyhdr}

\pagestyle{fancy}
\fancyhf{}

%\fancyhead[R]{\thepage}
\fancyhead[R]{\leftmark}

%\fancyfoot[R]{%
%  \vspace*{25mm}\transparent{0.1}\includegraphics[height=1.9cm]{images/logo_upct.png}%
%}
% Portada y contraportada con logo UPCT
%\AddToShipoutPictureFG{%
%  \put(\LenToUnit{\paperwidth-3.5cm},1cm){%
%    \transparent{0.25}\includegraphics[height=1.2cm]{images/logo_upct.png}%
%  }%
%}

\renewcommand{\headrulewidth}{0pt}
\setlength{\headheight}{16pt}

\fancypagestyle{plain}{
  \fancyhf{}
  \fancyhead[R]{\thepage}
%\fancyfoot[R]{%
%  \vspace*{25mm}\transparent{0.1}\includegraphics[height=1.9cm]{images/logo_upct.png}%
%}
\AddToShipoutPictureFG{%
  \put(\LenToUnit{\paperwidth-3.5cm},1cm){%
    \transparent{0.25}\includegraphics[height=1.2cm]{images/logo_upct.png}%
  }%
}



  \renewcommand{\headrulewidth}{0pt}
}

%encabezados de capítulos
%-------------------------------
% Usamos titlesec para controlar el formato de los títulos
\usepackage{titlesec}

% Título de capítulo: número y texto en negrita, tamaño mayor
\titleformat{\chapter}[display]
  {\normalfont\huge\bfseries} % formato del título
  {\chaptername\ \thechapter} % etiqueta del capítulo
  {20pt}                         % separación entre número y título
  {\Huge}                        % formato del texto del título

% Secciones también en negrita, con tamaño apropiado
\titleformat{\section}
  {\normalfont\Large\bfseries}{\thesection}{1em}{}

%-------------------------------
% Comienzo del documento
%-------------------------------


\begin{document}

\title{Estudio de P4 y su papel en la telemetría de red en banda para las redes 5G y beyond}
\author{Edinson Montero Beltre}
\date{\today}

\maketitle
\tableofcontents

% Prefacio
\chapter*{Dedicatory}
A quienes me apoyaron en este proyecto.
\addcontentsline{toc}{chapter}{Dedicatoria}
\include{preface/3.TableOfContents}
\include{preface/4.TableOfFigures}
\include{preface/5.TableOfTables}
\chapter*{Acrónimos}
\addcontentsline{toc}{chapter}{Acrónimos}

\noindent
\begin{tabular}{p{2cm}p{12cm}}
\toprule
\textbf{Acrónimo} & \textbf{Significado} \\
\midrule
5G & Fifth Generation (Quinta Generación) \\
5GC & 5G Core (Núcleo de Red 5G) \\
API & Application Programming Interface (Interfaz de Programación de Aplicaciones) \\
DNS & Domain Name System (Sistema de Nombres de Dominio) \\
INT & In-band Network Telemetry (Telemetría en Banda) \\
IP & Internet Protocol (Protocolo de Internet) \\
MIMO & Multiple Input Multiple Output (Entrada Múltiple Salida Múltiple) \\
NFV & Network Functions Virtualization (Virtualización de Funciones de Red) \\
P4 & Programming Protocol-Independent Packet Processors \\
QoS & Quality of Service (Calidad de Servicio) \\
SDN & Software-Defined Networking (Redes Definidas por Software) \\
TCP & Transmission Control Protocol (Protocolo de Control de Transmisión) \\
TLS & Transport Layer Security (Seguridad de la Capa de Transporte) \\
UDP & User Datagram Protocol (Protocolo de Datagrama de Usuario) \\
UE & User Equipment (Equipo de Usuario) \\
UL & Uplink (Enlace Ascendente) \\
UMTS & Universal Mobile Telecommunications System (Sistema Universal de Telecomunicaciones Móviles) \\
UPF & User Plane Function (Función del Plano de Usuario) \\
URLLC & Ultra-Reliable Low-Latency Communications (Comunicaciones de Ultra Fiabilidad y Baja Latencia) \\
\bottomrule
\end{tabular}

 

% Capítulos
\chapter{Introducción}
\section{Contexto general de las redes 5G y beyond}
Las redes de quinta generación (5G) representan un avance significativo en la evolución de las telecomunicaciones móviles, ofreciendo mayores velocidades, menor latencia y una capacidad mejorada para conectar dispositivos masivos.
 Con la creciente demanda de servicios en tiempo real y aplicaciones críticas, como la realidad aumentada, los vehículos autónomos y la telemedicina, en donde la latencia y la fiabilidad son esenciales debido a que cualquier retraso puede tener consecuencias graves, la necesidad de una visibilidad y monitoreo efectivos de la red se ha vuelto crucial. En esta evolución, las bajas latencias son cruciales para los nuevos servicios en la actualidad y los futuros, incluyendo la cuarta revolucion industrial.
%---------------------------------------
 \section{Motivación del estudio}
La complejidad y dinamismo de las redes 5G y superiores, plantean desafíos significativos para la gestión y el monitoreo de la red. Los métodos tradicionales de telemetría, como SNMP y NetFlow, a menudo no proporcionan la granularidad y la rapidez necesarias para detectar y 
resolver problemas en tiempo real otorgando una visibilidad de extremo a extremo sobre el estado de la red de forma precisa. La telemetría en banda (In-Band Network Telemetry, INT) emerge como una solución prometedora para abordar estas limitaciones, permitiendo la recopilación de datos detallados directamente desde los paquetes que atraviesan la red. Este estudio se centra en explorar el papel de P4, un lenguaje de programación para definir el comportamiento de los dispositivos de red, en la implementación y optimización de soluciones de telemetría en banda para redes 5G y beyond.
%---------------------------------------
\section{Problemática en la visibilidad y monitoreo de redes 5G y beyond}
A medida que las redes 5G se vuelven más complejas, la visibilidad y el monitoreo efectivos se convierten en desafíos críticos. La diversidad de servicios y la naturaleza dinámica del tráfico generan dificultades para identificar cuellos de botella, latencias elevadas y otros problemas de rendimiento. Además, la necesidad de cumplir con estrictos requisitos de calidad de servicio (QoS) y experiencia del usuario (QoE) exige soluciones de monitoreo que puedan adaptarse rápidamente a las condiciones cambiantes de la red. La problemática radica en cómo implementar mecanismos de telemetría que sean capaces de proporcionar datos precisos y en tiempo real sin introducir una sobrecarga significativa en la red.
%---------------------------------------
\section{Hipótesis y preguntas de investigación}
La hipótesis central de este estudio es que la implementación de telemetría en banda utilizando P4 puede mejorar significativamente la visibilidad y el monitoreo de las redes 5G y beyond, permitiendo una gestión más eficiente y una mejor calidad de servicio. Las preguntas de investigación que guían este estudio incluyen:
\begin{itemize}
    \item ¿Cómo puede P4 facilitar la implementación de soluciones de telemetría en banda en redes 5G y beyond?
    \item ¿Qué beneficios específicos ofrece la telemetría en banda en comparación con los métodos tradicionales de monitoreo?
    \item ¿Cuáles son los desafíos y limitaciones asociados con el uso de P4 para telemetría en banda en entornos 5G y beyond?
    \item ¿Cómo afecta la telemetría en banda al rendimiento general de la red y a la experiencia del usuario?
\end{itemize}
%---------------------------------------
\section{Objetivos generales y específicos}
El objetivo general de este estudio es evaluar el papel de P4 en la implementación de telemetría en banda para mejorar la visibilidad y el monitoreo de las redes 5G y beyond. Los objetivos específicos incluyen:
\begin{itemize}
    \item Analizar las capacidades de P4 para definir y programar el comportamiento de los dispositivos de red en el contexto de la telemetría en banda, específicamente INT-MD.
    \item Diseñar e implementar un prototipo de telemetría en banda (INT-MD) utilizando P4 en un entorno de red 5G SA simulado.
    \item Evaluar el rendimiento y la efectividad de la solución propuesta en términos de visibilidad, latencia y sobrecarga de la red.
    \item Identificar los desafíos y limitaciones encontrados durante la implementación y proponer posibles soluciones o mejoras.
\end{itemize}
%---------------------------------------
\section{Metodología de trabajo}
Este estudio adoptará un enfoque experimental y analítico para investigar el papel de P4 en la telemetría en banda para redes 5G y beyond. La metodología incluirá las siguientes etapas:
\begin{itemize}
    \item Revisión bibliográfica: Se realizará una revisión exhaustiva de la literatura existente sobre telemetría en banda, P4 y redes 5G y beyond para establecer un marco teórico sólido.
    \item Diseño del prototipo: Se diseñará un prototipo de telemetría en banda utilizando P4, definiendo las tablas y acciones necesarias para insertar metadatos INT-MD en el tráfico N2 (SCTP) y N3 (GTP-U).
    \item Implementación: El prototipo se implementará en un entorno emulado usando herramientas como GNS3 y VMware workstation que contará con Routers Cisco y switches P4 (BMv2), los cuales serán programados para soportar INT-MD y por ultimo, un servidor sink para la recolección y análisis de los datos de telemetría usando influxDB y Grafana para representar los datos. Este prototipo integrará un core 5G SA básico y una red de distribución programable con P4.
    \item Evaluación: Se llevarán a cabo pruebas para evaluar el rendimiento del prototipo, midiendo métricas como la latencia, la sobrecarga de la red y la precisión de los datos recopilados, asi como el impacto de la telemetría en banda en la calidad del servicio.
    \item Análisis de resultados: Los resultados obtenidos se analizarán críticamente para identificar beneficios, desafíos y áreas de mejora.
\end{itemize}
%---------------------------------------
\section{Estructura del documento}
El documento se estructura en varios capítulos que abordan diferentes aspectos del estudio:
\begin{itemize}
    \item Capítulo 1: \textbf{Introducción} - Presenta el contexto, la motivación, los objetivos y la metodología del estudio.
    \item Capítulo 2: \textbf{Redes 5G} - Proporciona una visión general de las redes 5G, sus características y desafíos.
    \item Capítulo 3: \textbf{Telemetría en Redes 5G} - Explora la necesidad de telemetría, los métodos tradicionales y la telemetría en banda (INT).
    \item Capítulo 4: \textbf{Fundamentos de P4} - Describe el lenguaje P4, su arquitectura y capacidades relevantes para la telemetría.
    \item Capítulo 5: \textbf{Entorno de Desarrollo} - Detalla el entorno utilizado para implementar y probar el prototipo de telemetría en banda.
    \item Capítulo 6: \textbf{Resultados} - Presenta los resultados obtenidos durante las pruebas del prototipo.
    \item Capítulo 7: \textbf{Discusión} - Analiza críticamente los resultados, comparándolos con el estado del arte y discutiendo beneficios y desafíos.
    \item Capítulo 8: \textbf{Conclusiones y Trabajo Futuro} - Resume las conclusiones principales, contribuciones, limitaciones y propone líneas futuras de investigación.
\end{itemize}
\section{Resumen de capítulos}
Cada capítulo del documento se enfoca en aspectos específicos del estudio, proporcionando una comprensión integral del papel de P4 en la telemetría en banda para redes 5G. A lo largo del documento, se integran conceptos teóricos con aplicaciones prácticas, culminando en un análisis crítico de los resultados obtenidos y su relevancia para el campo de las telecomunicaciones móviles.

\chapter{Fundamentos de las Redes 5G y Beyond}

\section{Evolución de tecnologías móviles}
Las redes móviles han evolucionado significativamente desde su inicio, pasando por varias generaciones que han mejorado la capacidad, velocidad y latencia de las comunicaciones inalámbricas. pasando a ser una pieza clave en la infraestructura de telecomunicaciones global, impactando positivamente en diversos sectores como la industria, la salud y el entretenimiento, entre otros.
\subsection{De 1G a 4G: Hitos clave}
\item \textbf{1G}: Introducción de la comunicación analógica.
\item \textbf{2G}: Digitalización de la voz y servicios básicos de datos (SMS).
\item \textbf{3G}: Introducción de datos móviles y servicios multimedia.
\item \textbf{4G}: Redes IP y mayor velocidad de datos.
\item \textbf{LTE y LTE-Advanced}: Mejoras en eficiencia espectral y latencia.

\subsection{De 4G a 5G SA}
\item \textbf{5G Non-Standalone (NSA)}: Uso combinado de 4G y 5G para una transición suave. Aumentando la capacidad y velocidad de la red.
\item \textbf{5G Standalone (SA)}: Arquitectura completamente nueva basada en servicios y virtualización, permitiendo nuevas funcionalidades y mejoras en latencia y confiabilidad.
 
\subsection{5G Advanced y Beyond-5G}
\item \textbf{5G Advanced}: Mejoras en IA, eficiencia energética y soporte para nuevas aplicaciones, como XR y comunicaciones vehiculares.
\item \textbf{Beyond-5G}: Investigación en tecnologías emergentes como comunicaciones cuánticas y redes holográficas, con miras a la futura generación 6G.

\subsection{Tendencias hacia 6G}
\item \textbf{Visión de 6G}: Redes ultra confiables, latencia casi nula y capacidades de inteligencia artificial integradas de manera nativa.
\item \textbf{Tecnologías emergentes}: Comunicaciones terahertz, redes auto-organizadas y computación en el borde (Edge Computing).
\item \textbf{Aplicaciones futuras}: Realidad extendida (XR), ciudades inteligentes (Smart Cities) y redes de sensores masivos (IoT masivo).
\item \textbf{Desafíos}: Seguridad, privacidad y sostenibilidad ambiental.

%----------------------------------------
\section{Características de RAN para 5G y beyond}
En 5G (Release 15) el UE puede conectarse a la red mediante dos tipos de acceso: LTE (E-UTRA) y NR (New Radio). La red Core puede ser EPC (Evolved Packet Core) o 5GC (5G Core), dependiendo de si la red es NSA o SA, respectivamente. Posteriormente, esto se amplió para que ambas celdas puedan pertenecer a 5G NR, en cuyo caso la CN es exclusivamente 5G Core. Estas diversas opciones se agrupan en el término Multi-Radio Dual Connectivity (MR-DC). MR-DC es una generalizacion de Intra-E-UTRA Dual Connectivity (EN-DC) y E-UTRA-NR Dual Connectivity (NE-DC) que permite que un UE se conecte simultáneamente a dos nodos de acceso, que pueden ser E-UTRA o NR. En MR-DC, un nodo de acceso actúa como el nodo maestro o master node (MN) y el otro como el nodo secundario o secondary node (SN). El MN es responsable de la señalización y el control, mientras que el SN se utiliza principalmente para la transferencia de datos.



\section{Arquitectura 5G Non Standalone (NSA) }
La arquitectura 5G Non-Standalone (NSA) es una configuración de red que permite la coexistencia y colaboración entre las tecnologías 4G LTE y 5G NR. En esta arquitectura, la red 4G LTE actúa como la red principal para la señalización y el control, mientras que la red 5G NR se utiliza principalmente para el transporte de datos de alta velocidad. Esta configuración permite a los operadores de red aprovechar la infraestructura existente de LTE mientras implementan nuevas capacidades de NR, facilitando una transición más suave hacia las redes 5G completas.

Desde una perspectiva operacional, la arquitectura NSA permite a los proveedores de servicios móviles activar capacidades 5G de forma gradual sin necesidad de reemplazar completamente la infraestructura de núcleo de red existente. El despliegue de NSA comienza típicamente con la instalación de nuevas estaciones base NR que coexisten con las estaciones base LTE existentes. Los dispositivos del usuario (UE) que soportan tanto LTE como NR pueden conectarse simultáneamente a ambas redes, aprovechando la mejor señal disponible para la señalización de control a través de LTE y utilizando NR para el tráfico de datos cuando está disponible. Este enfoque dual proporciona beneficios inmediatos de rendimiento sin requerir una arquitectura de núcleo completamente nueva, lo que reduce significativamente los costos de inversión durante la transición.

La arquitectura NSA introduce el concepto de "Dual Connectivity", donde un dispositivo móvil mantiene conexiones simultáneas a dos estaciones base: una primaria (master) y una secundaria (secundaria). En el caso más común de NSA, la estación base LTE actúa como el nodo maestro, proporcionando la conexión primaria y controlando el plano de control, mientras que la estación base NR actúa como un nodo secundario dedicado al tráfico de datos de alta velocidad. Esta separación de funciones permite que los operadores optimicen el uso del espectro radioeléctrico, permitiendo que LTE continúe manejando eficientemente la señalización de control y la cobertura amplia, mientras que NR proporciona capacidad adicional de datos donde la densidad de usuarios es más alta.

Desde el punto de vista del usuario final, la arquitectura NSA ofrece una experiencia mejorada en comparación con las redes LTE puras. Los usuarios pueden experimentar velocidades de descarga más altas (potencialmente en el rango de gigabits por segundo), menor latencia en aplicaciones específicas de datos, y mejor eficiencia general de la red. Sin embargo, las ventajas están limitadas principalmente al tráfico de datos, ya que los servicios de señalización y control siguen estando limitados por las especificaciones de LTE. Para casos de uso que requieren baja latencia extrema o requisitos ultra confiables (como comunicaciones críticas), la arquitectura NSA puede no ser suficiente, lo que subraya la necesidad eventual de migrar a 5G SA para alcanzar todas las capacidades de 5G.

\subsection{Opciones de arquitectura 3GPP 5G NSA}
El trabajo realizado por el 3GPP sobre la arquitectura de la red 5G dio como resultado un conjunto de alternativas arquitectónicas, fundamentadas en tres decisiones clave del propio 3GPP. Este estudio esta documentado en el reporte técnico 3GPP TR 23.799 \cite{3gpp.23.799}, donde se describen las diferentes opciones arquitectónicas para la implementación de redes 5G NSA. A continuación, se presentan las principales opciones:
\begin{itemize}
    \item \textbf{Opción 1}: Arquitectura basada en LTE/EPC, donde el Core de la red sigue siendo el EPC y se utiliza NR como acceso adicional.
    \item \textbf{Opción 2}: Arquitectura basada en 5G Core (5GC), donde tanto el acceso como el Core de la red son 5G (SA).
    \item \textbf{Opción 3}: Arquitectura híbrida que combina elementos de LTE/EPC y 5GC, permitiendo una transición gradual hacia 5G.
\end{itemize}   
El hecho de que la red de acceso LTE y NR puedan coexistir en una misma red 5G NSA, permite a los operadores de red aprovechar la infraestructura existente de LTE mientras implementan nuevas capacidades de NR. Por lo que la red LTE (RAN) tiene dos formas de conectarse con el Core (Core):
\begin{itemize}
    \item \textbf{Conexión mediante S1 al EPC}: En esta configuración, la red LTE se conecta al EPC a través de la interfaz S1, mientras que la red NR se conecta al EPC a través de una interfaz adaptada. Esta opción permite una integración más sencilla con la infraestructura LTE existente.
    \item \textbf{Conexión mediante N2/N3 al Core 5GC}: En esta configuración, la red LTE se conecta al Core 5GC a través de las interfaces N2 y N3, mientras que la red NR también se conecta al 5GC. Esto permite una integración más estrecha entre las redes LTE y NR, facilitando la gestión y el control de la red.
\end{itemize}
Teniendo en cuenta que RAN de LTE y NR pueden coexistir en una misma red 5G NSA, existen cuatro formas de implementar LTE y/o NR. \cite{3gpp.23.799}:
\begin{itemize}
    \item \textbf{Opción 1}: Solo LTE para todo el trafico de datos y señalización.
    \item \textbf{Opción 2}: Solo NR para todo el trafico de datos y señalización.
    \item \textbf{Opción 3}: Una combinación de LTE y NR donde LTE tiene la mayor cobertura y se utiliza para señalización, mientras que LTE y NR se utilizan para tráfico de datos.
    \item \textbf{Opción 4}: Una combinación de LTE y NR donde NR tiene la mayor cobertura y se utiliza para señalización, mientras que LTE y NR se utilizan para tráfico de datos.
\end{itemize}
Considerando la integración de 2 Core diferentes (EPC y 5GC) en una red 5G NSA, existen 8 formas de implementar la arquitectura 5G NSA. \cite{3gpp.sp.160455}.

%-------Imagen Figura 2.1-------
\begin{figure}[h]
    \centering
    \includegraphics[width=1.0\textwidth]{images/combinacion-ran.png}
    \caption{Posibles combinaciones de RAN y Core en una red 5G NSA \cite{3gpp.sp.160455}.}
    \label{fig:5g-combinations}
\end{figure}
%-------------------------------
En la figura \ref{fig:5g-combinations} se muestran las diferentes combinaciones posibles de RAN y Core en una red 5G NSA 4 × 2 = 8. Las cuales son:
\begin{itemize}
    \item \textbf{Opción 1}: RAN LTE conectada al EPC. (Solo LTE)
    \item \textbf{Opción 2}: RAN LTE conectada al 5GC. (Solo LTE)
    \item \textbf{Opción 3}: RAN LTE y NR conectadas al EPC. (LTE para señalización y datos, NR para datos)
    \item \textbf{Opción 4}: RAN LTE y NR conectadas al 5GC. (LTE para señalización y datos, NR para datos)
    \item \textbf{Opción 5}: RAN NR y LTE conectadas al EPC. (NR para señalización y datos, LTE para datos)
    \item \textbf{Opción 6}: RAN NR y LTE conectadas al 5GC. (NR para señalización y datos, LTE para datos)
    \item \textbf{Opción 7}: RAN NR conectada al EPC. (Solo NR)
    \item \textbf{Opción 8}: RAN NR conectada al 5GC. (Solo NR)
\end{itemize}
Las opciones 3, 4, 5 y 6 son las que permiten la coexistencia de LTE y NR en la misma red 5G NSA, aprovechando las capacidades de ambas tecnologías para mejorar la cobertura y el rendimiento de la red. Estas opciones representan diferentes estrategias de integración que los operadores pueden adoptar según sus objetivos de despliegue y requisitos de infraestructura.

En las opciones 3 y 4, LTE actúa como la tecnología principal para la señalización de control y la cobertura amplia, mientras que NR se despliega como una tecnología complementaria para proporcionar capacidad adicional de datos de alta velocidad en áreas densamente pobladas. Esta configuración permite a los operadores mantener una cobertura LTE extendida (que puede alcanzar kilómetros de distancia) mientras aprovechan las bandas de frecuencia más altas de NR (donde la propagación es más limitada) para aumentar la capacidad de datos. Las opciones 3 y 4 difieren en el núcleo de red utilizado: la opción 3 utiliza el EPC tradicional de 4G, lo que requiere menos cambios en la infraestructura existente, mientras que la opción 4 utiliza el 5GC, permitiendo una integración más profunda con las funciones de red 5G.

Las opciones 5 y 6 invierten el rol de las tecnologías, donde NR se convierte en la tecnología primaria para la señalización y el control, mientras que LTE actúa como una tecnología secundaria para datos adicionales. Este enfoque es útil en escenarios donde los operadores han desplegado ampliamente NR y desean optimizar la utilización del espectro LTE existente o en regiones donde LTE proporciona una cobertura superior. Similar a las opciones anteriores, la opción 5 utiliza EPC como núcleo de red, mientras que la opción 6 utiliza 5GC, proporcionando mayor flexibilidad en la arquitectura de red.

La selección entre estas opciones (3, 4, 5 o 6) depende de varios factores estratégicos: la cobertura existente de LTE en la región, la disponibilidad de espectro NR, la madurez de la infraestructura 5GC disponible, los objetivos de calidad de servicio requeridos para diferentes tipos de aplicaciones, y las inversiones ya realizadas en infraestructura de red anterior. Los operadores típicamente comienzan con las opciones 3 o 4 (donde LTE es primario) para aprovechar su cobertura establecida, y gradualmente evolucionan hacia las opciones 5 o 6 (donde NR es primario) conforme la cobertura de NR se expande y se vuelve más dominante en la red.
%-----------------------------------------












\section{Arquitectura 5G Standalone (SA)}
\subsection{Visión general de la arquitectura 5G SA}
La arquitectura 5G Standalone (SA) representa una evolución significativa en comparación con las generaciones anteriores de redes móviles. A diferencia de las implementaciones Non-Standalone (NSA), que dependen de la infraestructura existente de 4G LTE, la arquitectura SA está diseñada desde cero para aprovechar al máximo las capacidades de la tecnología 5G. Esta arquitectura se basa en una serie de principios clave que permiten una mayor flexibilidad, eficiencia y capacidad para soportar una amplia gama de servicios y aplicaciones. Entre los aspectos más destacados de la arquitectura 5G SA se encuentran:
\begin{itemize}
    \item \textbf{Red basada en servicios}: La arquitectura 5G SA adopta un enfoque basado en servicios, donde las funciones de red se implementan como servicios independientes que pueden ser orquestados y gestionados de manera flexible.
    \item \textbf{Virtualización y desagregación}: La arquitectura permite la virtualización de funciones de red (NFV) y la desagregación de hardware y software, lo que facilita la implementación en entornos de nube y mejora la escalabilidad.
    \item \textbf{Separación del plano de control y plano de usuario}: Esta separación permite una gestión más eficiente del tráfico y una mejor calidad de servicio (QoS) para diferentes tipos de aplicaciones.
    \item \textbf{Soporte para nuevas tecnologías}: La arquitectura 5G SA está diseñada para integrar tecnologías emergentes como la inteligencia artificial (IA), el edge computing y la Internet de las cosas (IoT).
\end{itemize}

\subsection{Modos de implementación de 5G SA}
Existen tres modos principales de implementación de redes 5G NSA, que permiten la coexistencia de tecnologías LTE y NR en una misma red. Estos modos son:
\begin{itemize}
    \item \textbf{NGEN-DC (NG-RAN E-UTRA–NR Dual Connectivity)}: el UE (User Equipment) se conecta simultáneamente a una estación base LTE (eNodeB) como master node (MN) y a una estación base NR (gNodeB) como secondary node (SN), utilizando el EPC como Core de la red. El gNodeB (en-gNB) se puede conectar al EPC mediante la interface S1-U o x2-U para el plano de usuario y la interfaz S1-MME o X2-C para el plano de control.
    \item \textbf{NE-DC (NR–E-UTRA Dual Connectivity)}: el UE se conecta simultáneamente a una estación base NR como master node (MN) y a una estación base LTE como secondary node (SN), utilizando el 5GC como Core de la red.
    \item \textbf{NR-DC (NR–NR Dual Connectivity)}: el UE se conecta simultáneamente a dos estaciones base NR, una como master node (MN) y la otra como secondary node (SN), utilizando el 5GC como Core de la red.
\end{itemize}
La descripcion de DC se encuentra en el documento 3GPP TS 37.340 \cite{3gpp.ts.37.340}.




\subsection{Perspectivas en 5GC}
La arquitectura del núcleo de red 5G (5GC) introduce una serie de innovaciones y mejoras en comparación con las arquitecturas de núcleo de red anteriores, como el EPC utilizado en 4G. Estas mejoras están diseñadas para soportar las demandas crecientes de conectividad, velocidad y baja latencia que caracterizan a las redes 5G.

\section{Arquitectura basada en servicios (SBA)}
La diferencia más destacada en 5G frente a las arquitecturas 3GPP anteriores es la adopción del concepto de interfaces basadas en servicios \textbf{(SBA)} . Esto implica que las funciones de red que contienen la lógica y los mecanismos para procesar los flujos de señalización ya no se conectan mediante interfaces punto a punto, sino que \textbf{exponen sus capacidades como servicios} accesibles para otras funciones de red. En cada intercambio, una función actúa como \textbf{consumidora de servicios} y la otra como \textbf{proveedora de dichos servicios}. \ref{fig:5g-sba} muestra la arquitectura 5GC basada en SBA.

%-------Imagen Figura 2.2-------
\begin{figure}[H]
    \centering
    \includegraphics[width=1.0\textwidth]{images/sba.png}
    \caption{Arquitectura 5GC basada en interfaces basadas en servicios. \cite{oreillyf2021}.}
    \label{fig:5g-sba}
\end{figure}
%-------------------------------

Las funciones de red en 5GC se comunican entre sí a través de una interfaz común llamada \textbf{Service-Based Interface (SBI)}. Esta interfaz utiliza protocolos estándar como HTTP/2 y RESTful APIs para facilitar la comunicación entre las funciones de red. La adopción de SBA permite una mayor flexibilidad y escalabilidad en la arquitectura de la red, ya que las funciones pueden ser desarrolladas, desplegadas y actualizadas de manera independiente. Además, esta arquitectura facilita la integración con tecnologías emergentes como la virtualización de funciones de red (NFV) y la computación en la nube, permitiendo a los operadores de red adaptarse rápidamente a las demandas cambiantes del mercado y ofrecer nuevos servicios de manera más eficiente.

La arquitectura SBA también se puede representar como punto a punto, donde cada función de red se conecta directamente con las demás funciones que requieren sus servicios, utilizando la interfaz SBI para la comunicación. Como se muestra en la figura \ref{fig:sbi_p2p}.
%-------Imagen Figura 2.3-------
\begin{figure}[H]
    \centering
    \includegraphics[width=1.0\textwidth]{images/sbi_p2p.png}
    \caption{Arquitectura 5GC con interfaces punto a punto. \cite{oreillyf2021}.}
    \label{fig:sbi_p2p}
\end{figure}
%-------------------------------

\subsection{Interfaces HTTP REST}
En la arquitectura 5G Core (5GC), las funciones de red se comunican entre sí utilizando una interfaz basada en servicios conocida como Service-Based Interface (SBI). Esta interfaz utiliza el protocolo HTTP/2 junto con RESTful APIs para facilitar la comunicación y el intercambio de información entre las diferentes funciones de red. A continuación, se describen los aspectos clave de las interfaces HTTP REST en 5GC:
\begin{itemize}
    \item \textbf{Protocolo HTTP/2}: 5GC utiliza HTTP/2 como el protocolo de transporte para las comunicaciones entre funciones de red. HTTP/2 ofrece varias ventajas sobre su predecesor, HTTP/1.1, incluyendo una mayor eficiencia en la multiplexación de solicitudes, compresión de encabezados y reducción de la latencia, lo que es crucial para las aplicaciones de baja latencia en 5G.
    \item \textbf{RESTful APIs}: Las funciones de red en 5GC exponen sus capacidades a través de RESTful APIs, que son interfaces basadas en principios REST (Representational State Transfer). Estas APIs permiten a las funciones de red interactuar de manera sencilla y estandarizada, utilizando métodos HTTP como GET, POST, PUT y DELETE para realizar operaciones sobre los recursos.
    \item \textbf{Formato de datos JSON}: La comunicación entre funciones de red a través de las RESTful APIs generalmente utiliza JSON (JavaScript Object Notation) como formato de datos para el intercambio de información. JSON es ligero y fácil de leer, lo que facilita la integración entre diferentes sistemas y tecnologías.
    \item \textbf{Seguridad}: La seguridad es un aspecto crítico en las comunicaciones entre funciones de red. En 5GC, se implementan mecanismos de autenticación y autorización para garantizar que solo las funciones autorizadas puedan acceder a los servicios expuestos a través de las RESTful APIs. Además, se utilizan protocolos seguros como TLS (Transport Layer Security) para proteger la integridad y confidencialidad de los datos transmitidos.
    \item \textbf{Escalabilidad y flexibilidad}: La adopción de interfaces HTTP REST permite una mayor escalabilidad y flexibilidad en la arquitectura 5GC. Las funciones de red pueden ser desarrolladas, desplegadas y actualizadas de manera independiente, lo que facilita la adaptación a las demandas cambiantes del mercado y la incorporación de nuevas tecnologías.
\end{itemize}

\section{Componentes de 5G Core (5GC)}
En 5G Core (5GC), los componentes tienes funciones específicas a diferencia de las arquitecturas anteriores, las cuales combinaban algunas funciones como por ejemplo, el MME y el SGW en una sola entidad llamada AMF.

\subsection{Componentes principales de 5GC}
\subsubsection{NRF}
El Network Repository Function (NRF) es un componente clave en la arquitectura 5G Core (5GC) que actúa como un repositorio centralizado para la gestión y descubrimiento de servicios de red. Su función principal es mantener un registro actualizado de todas las funciones de red disponibles en la red 5G, permitiendo que otras funciones de red puedan descubrir y comunicarse con ellas de manera eficiente. acorde a \cite{3gpp.ts.23.501}.
Algunas de las funciones principales del NRF incluyen:

















\subsection{AMF}
El Access and Mobility Management Function (AMF) es responsable de la gestión de la movilidad y el acceso de los dispositivos de usuario (UE) en la red 5G. Sus funciones principales incluyen:
\item Gestión de la conexión y autenticación del UE.
\item Manejo de la movilidad del UE, incluyendo el seguimiento y la gestión de la ubicación.
\item Coordinación con otras funciones de red para garantizar una experiencia de usuario fluida durante los cambios de celda y las transiciones entre diferentes tecnologías de acceso.
\item Gestión de la seguridad y la integridad de las comunicaciones entre el UE y la red.
\item Coordinación con el Session Management Function (SMF) para la gestión de sesiones de datos.
\subsection{PCF}
El Policy Control Function (PCF) es responsable de la gestión y aplicación de políticas en la red 5G. Sus funciones principales incluyen:
\item Definición y aplicación de políticas de calidad de servicio (QoS) para diferentes tipos de tráfico.
\item Gestión de políticas de acceso y control de recursos en función de las condiciones de la red y las necesidades del usuario.
\item Coordinación con otras funciones de red, como el AMF y el SMF, para garantizar que las políticas se apliquen de manera coherente en toda la red.



\subsection{SMF}
\subsection{UPF}
\subsection{AUSF}
\subsection{UDM}
\subsection{PCF}

\subsection{NSSF}
\subsection{NEF}
\subsection{AF}
\subsection{Funciones del plano de control y plano de usuario}
\subsection{Interfaces clave: N1, N2, N3, N4, N6, N9}
\subsection{NGAP y N2}
\subsection{NGRAN y separación CP/UP}

\section{Protocolos relevantes en 5G}
\subsection{NGAP}
\subsection{SCTP en 5G}
\subsection{GTP-U}
\subsection{PFCP}

\section{QoS en 5G}
\subsection{Conceptos de QFI, 5QI y políticas de flujo}
\subsection{Programación del tráfico en UPF}
\subsection{QoS-based routing y desafíos actuales}
\subsection{Relación entre QoS y telemetría}

\chapter{Telemetría y QoS en 5G}
En 5GC, la telemetria juega un papel crucial en la monitorización y gestión del rendimiento de la red, especialmente en el contexto de las demandas crecientes de servicios como URLLC (Ultra-Reliable Low-Latency Communications), eMBB (enhanced Mobile Broadband) y mMTC (massive Machine Type Communications). La telemetría en banda (In-Band Network Telemetry, INT) permite la recopilación de datos en tiempo real directamente desde los paquetes que atraviesan la red, proporcionando una visibilidad detallada del estado de la red y facilitando la detección y resolución de problemas.

\section{Desafíos de Telemetría en Redes 5G}
\subsection{Limitaciones de mecanismos tradicionales (SNMP, NetFlow, sFlow)}
\subsection{Requisitos de visibilidad en URLLC, eMBB y mMTC}

\section{In-Band Network Telemetry (INT)}
\subsection{Concepto y motivación}
\subsection{Arquitectura INT: source, transit y sink}
\subsection{Metadatos INT-MD y su estructura}
\subsection{INT vs telemetría out-of-band}
%\section{Telemetría aplicada en 5G}
%\subsection{Relevancia de INT en el plano de control N2}
%\subsection{Relevancia de INT en el plano de usuario N3}
%\subsection{Métricas clave para tráfico GTP-U}
%\subsection{Desafíos y limitaciones en redes móviles}
%--- QoS en 5G ---
\section{Calidad de Servicio en 5G}
\section{QoS basado en 5G Flujo}
\subsection{Definición de flujo en 5G}
\subsection{Identificación y clasificación de flujos}
\subsection{Mecanismos de gestión de QoS por flujo}
\section{Señalización de QoS en 5G}
\section{Caracteristicas de QoS en 5G}
\chapter{Fundamentos de P4 y Programación del Plano de Datos}

\section{Introducción a P4}
\subsection{Historia y evolución}
\subsection{Modelo de arquitectura PISA}
\subsection{P4\_14 vs P4\_16}

\section{Herramientas y ecosistema P4}
\subsection{BMv2 como switch software}
\subsection{P4C: compilador}
\subsection{P4Runtime: control del data plane}

\section{Implementación de INT en P4}
\subsection{Definición de cabeceras INT}
\subsection{Parsing y deparsing en BMv2}
\subsection{Inyección hop-by-hop de metadatos}
\subsection{Limitaciones del pipeline P4}

\chapter{Entorno de Desarrollo Implementado}

\section{Arquitectura general del sistema}
A diferencia de las tecnologias pasadas, 5G esta diseñada para ser facilmente desplegada en entornos virtualizados y basados en contenedores, lo que permite una mayor flexibilidad y escalabilidad. En este proyecto, se ha implementado un entorno de desarrollo que emula una red 5G Standalone (SA), virtualizando sus componentes principales mediante Docker y orquestando el Core con Docker Compose.
El entorno de desarrollo se ha desplegado utilizando GNS3, GNS3 VM y VMware Workstation para crear una infraestructura base o underlay sobre la cual se han implementado otras máquinas virtuales  que alojan los contenedores Docker. Esta infraestructura virtual proporciona la conectividad necesaria entre los componentes y permite emular una red de un MNO (Mobile Network Operator) real, con componentes tanto de red y VFs (Virtualized Functions) como de 5G. Tambien cabe destacar que para los componentes de 5G se empleó el projecto open source free5gc \cite{free5gc}, el cual permite desplegar un core 5G SA completo en un entorno virtualizado con Docker. La siguiente figura muestra la arquitectura general del entorno de desarrollo implementado, destacando la integración entre GNS3, Docker y la máquina virtual en vmware \ref{fig:diseno_sistema}.
 \begin{figure}[H]
    \centering
    \includegraphics[width=0.9\textwidth]{images/diseno-sistema.png}
    \caption{Arquitectura general}
    \label{fig:diseno_sistema}
\end{figure}
\subsection{Diseño de la red 5G SA}
Con respecto a la red de transporte, se ha implementado utilizando switches programables P4 para insertar metadatos de telemetría en banda (INT-MD) en el tráfico N2 (SCTP) y N3 (GTP-U), estos switches estan construidos con P4 para el reenvio de trafico a nivel de L3. ademas de estos switches P4, tanto para el trafico entre los Nodos en el Core como en la comunicacion entre el NGRAN y R1, se ha empleado switches virtuales Open vSwitch (OVS) para gestionar gestionar los paquetes ARP. La siguiente figura es un diseño de alto nivel (HLD) quemuestra la topología de red implementada en GNS3 \ref{fig:topologia_red}.
\subsubsection{Diseño de alto nivel (HLD)}
\begin{figure}[H]
    \centering
    \includegraphics[width=1.0\textwidth]{images/Topo.png}
    \caption{Topología de red implementada en GNS3}
    \label{fig:topologia_red}
\end{figure}

\subsubsection{Diseño de bajo nivel (LLD)}
La anterior tabla \ref{tab:lld_containers} detalla la configuración de red de cada contenedor Docker, direcciones IP y MAC asignadas, el tipo de red utilizado (macvlan), etc. Con esta configuración, se asegura una correcta interconexión y funcionamiento del Core 5G SA dentro del entorno virtualizado.

\begin{table}[H]
\centering
\tiny
\setlength{\tabcolsep}{4pt}
\begin{tabular}{|c|c|c|c|c|c|c|c|c|}
\toprule
\textbf{NF} & \textbf{Direccion IP} & \textbf{Mascara} &\textbf{Direccion MAC} & \textbf{Tipo de Red} & \textbf{Alias} & \textbf{Puertos} & \textbf{BD} & \textbf{Deps.} \\
\midrule
db & 172.18.0.50 & /24 & 22:e5:63:00:49:85 & macvlan & db & 27017 & - & - \\
\midrule
NRF & 172.18.0.51 & /24 & 3a:e5:07:18:07:fb & macvlan & nrf.free5gc.org & 8000 & db & db \\
\midrule
AUSF & 172.18.0.52 & /24 & aa:61:e3:b4:07:2e & macvlan & ausf.free5gc.org & 8000 & - & nrf \\
\midrule
NSSF & 172.18.0.53 & /24 & 3e:74:95:c5:bc:4e & macvlan & nssf.free5gc.org & 8000 & - & nrf \\
\midrule
PCF & 172.18.0.54 & /24 & 72:b2:53:59:fc:ce & macvlan & pcf.free5gc.org & 8000 & - & nrf \\
\midrule
UDM & 172.18.0.55 & /24 & 9a:a1:ed:b3:60:8c & macvlan & udm.free5gc.org & 8000 & - & db, nrf \\
\midrule
UDR & 172.18.0.56 & /24 & ea:b1:9d:90:a4:d1 & macvlan & udr.free5gc.org & 8000 & db & db, nrf \\
\midrule
CHF & 172.18.0.57 & /24 & ee:e6:15:22:6f:ad & macvlan & chf.free5gc.org & 8000, 2122 & db & db, nrf, webui \\
\midrule
NEF & 172.18.0.58 & /24 & e2:66:f3:0f:e0:e2 & macvlan & nef.free5gc.org & 8000 & db & db, nrf \\
\midrule
WebUI & 172.18.0.59 & /24 & 2a:5c:48:5c:aa:a0 & macvlan & webui & 2121 (ext: 5000) & - & db, nrf \\
\midrule
AMF & 172.18.0.20 & /24 & f2:ff:cb:34:07:b5 & macvlan & amf.free5gc.org & 38412 (SCTP), 8000 & - & - \\
\midrule
SMF & 172.18.0.22 & /24 & ca:76:58:9c:8e:ec & macvlan & smf.free5gc.org & 8000 & - & - \\
\midrule
UPF & 172.18.0.24 & /24 & be:44:98:a4:e5:c4 & macvlan & upf.free5gc.org & 8000, 38412 (SCTP) & - & - \\
\midrule
ueransim & 172.18.10.3 & /29 & f6:b8:eb:4d:0f:3c & macvlan & ueransim.free5gc.org & 38412 (SCTP) & - & - \\
\bottomrule
\end{tabular}
\caption{Diseño de bajo nivel (LLD)}
\label{tab:lld_containers}
\end{table}


\section{Despliegue del Core 5G SA}
Como se pudo apreciar en la figura \ref{fig:diseno_sistema}, el Core 5G SA se ha desplegado utilizando contenedores Docker orquestados con Docker Compose dentro de maquinas virtuales alojadas en una maquina virtual principal (GNS3 VM). los componentes que interactuan mediante HTTP/2 se han desplegado en una VM con Ubuntu 20.04, mientras que el resto de componentes; AMF, SMF y UPF se han desplegado en maquinas virtuales separadas con la misma version de Ubuntu. Juntar los componentes que interactuan mediante HTTP/2 en una sola VM permite reducir la latencia y mejorar el rendimiento de las comunicaciones entre ellos.

La configuracion de red de cada contenedor Docker se realizó con la modalidad de red \textbf{macvlan}, detallada en la tabla \ref{tab:lld_containers}, que permite asignar una direccion IP en el mismo ranfo que la red fisica de la VM anfitriona. Esto facilita la comunicacion entre los contenedores y hace posible el establecimiento del enlace \textbf{N2} (SCTP) entre el NGRAN y el AMF.

\subsection{Configuracion de Core-Host}
Los detalles sobre la instalacion de Docker y Docker Compose no estan incluidos en este documento, pero se asume que el lector tiene conocimientos basicos sobre estas tecnologias. En caso contrario, puede consultar la documentacion oficial \cite{docker2024install}.
\subsubsection{Obtención del código fuente 5GC}
El primer paso para desplegar el Core 5G SA es clonar el repositorio oficial de free5gc desde GitHub como se muestra en la figura \ref{fig:clone-repo}.

\begin{figure}[H]
    \centering
    \includegraphics[width=0.9\textwidth]{images/clone-repo.png}
    \caption{Clonación del repositorio de free5gc}
    \label{fig:clone-repo}
\end{figure}

Luego de clonar el repositorio, se procede a clonar las bases de las Network Functions (NFs) adicionales que no vienen incluidas en el repositorio principal de free5gc, como se muestra en la figura \ref{fig:clonacion-nf}.
\begin{figure}[H]
    \centering
    \includegraphics[width=1.0\textwidth]{images/clonacion-nf.png}
    \caption{Clonación de base de NFs adicionales}
    \label{fig:clonacion-nf}
\end{figure}

Una vez clonado el repositorio principal y las bases de las NFs adicionales, se procede a complilar las NFs que en este caso, como es en el Core-Host, los compilamos todos como se puede ver en la figura \ref{fig:compilacion-nf}.
\begin{figure}[H]
    \centering
    \includegraphics[width=1.0\textwidth]{images/compilacion-nf.png}
    \caption{Compilación de NFs adicionales}
    \label{fig:compilacion-nf}
\end{figure}

\subsubsection{Configuracion de Docker Compose}
Luego de haberlos compliado, se procede a configurar cada NF de acuerdo con el diseño de bajo nivel (LLD) mostrado en la tabla \ref{tab:lld_containers}.
Como la VM Core-Host corre los contenedores UDR, UDM, PCF, NRF, AUSF, UDM y NSSF, DB, se despliegan creando un unico archivo Docker Compose llamado \textit{docker-compose-build-core-host.yaml}, el cual se mostrará contenedor por contenedor a continuacion:
\subsubsubsection{DB}

\begin{lstlisting}[style=yamlstyle, caption={Configuración del MongoDB en Docker Compose}, label=lst:docker-db]
services:
  db:
    container_name: mongodb
    image: mongo:3.6.8
    command: mongod --port 27017
    expose:
      - "27017"
    volumes:
      - dbdata:/data/db
    networks:
      macvlan_net:
        ipv4_address: 172.18.0.50
        mac_address: "22:e5:63:00:49:85"
        aliases:
          - db
\end{lstlisting}

\subsubsubsection{NRF}
\begin{lstlisting}[style=yamlstyle, caption={Configuración del NRF en Docker Compose}, label=lst:docker-nrf]
  free5gc-nrf:
    container_name: nrf
    build:
      context: ./nf_nrf
      args:
        DEBUG_TOOLS: "false"
    command: ./nrf -c ./config/nrfcfg.yaml
    expose:
      - "8000"
    volumes:
      - ./config/nrfcfg.yaml:/free5gc/config/nrfcfg.yaml
      - ./cert:/free5gc/cert
    environment:
      DB_URI: mongodb://db/free5gc
      GIN_MODE: release
    networks:
      macvlan_net:
        ipv4_address: 172.18.0.51
        mac_address: "3a:e5:07:18:07:fb"
        aliases:
          - nrf.free5gc.org
    extra_hosts:
     - "amf.free5gc.org:172.18.0.20"
     - "smf.free5gc.org:172.18.0.22"
     - "upf.free5gc.org:172.18.0.23"
    ports:
      - "8000"
      
    depends_on:
      - db
\end{lstlisting}

\subsubsubsection{AUSF}
\begin{lstlisting}[style=yamlstyle, caption={Configuración del AUSF en Docker Compose}, label=lst:docker-ausf]
  free5gc-ausf:
    container_name: ausf
    build:
      context: ./nf_ausf
      args:
        DEBUG_TOOLS: "false"
    command: ./ausf -c ./config/ausfcfg.yaml
    expose:
      - "8000"
    volumes:
      - ./config/ausfcfg.yaml:/free5gc/config/ausfcfg.yaml
      - ./cert:/free5gc/cert
    environment:
      GIN_MODE: release
    networks:
      macvlan_net:
        ipv4_address: 172.18.0.52
        mac_address: "aa:61:e3:b4:07:2e"
        aliases:
          - ausf.free5gc.org
    extra_hosts:
     - "amf.free5gc.org:172.18.0.20"
     - "smf.free5gc.org:172.18.0.22"
     - "upf.free5gc.org:172.18.0.23"

    depends_on:
      - free5gc-nrf
\end{lstlisting}
\subsubsubsection{NSSF}
\begin{lstlisting}[style=yamlstyle, caption={Configuración del NSSF en Docker Compose}, label=lst:docker-nssf]
  free5gc-nssf:
    container_name: nssf
    build:
      context: ./nf_nssf
      args:
        DEBUG_TOOLS: "false"
    command: ./nssf -c ./config/nssfcfg.yaml
    expose:
      - "8000"
    volumes:
      - ./config/nssfcfg.yaml:/free5gc/config/nssfcfg.yaml
      - ./cert:/free5gc/cert
    environment:
      GIN_MODE: release
    networks:
      macvlan_net:
        ipv4_address: 172.18.0.53
        mac_address: "3e:74:95:c5:bc:4e"
        aliases: 
          - nssf.free5gc.org
    extra_hosts:
     - "amf.free5gc.org:172.18.0.20"
     - "smf.free5gc.org:172.18.0.22"
     - "upf.free5gc.org:172.18.0.23"  
    depends_on:
      - free5gc-nrf
\end{lstlisting}
\subsubsubsection{PCF}
\begin{lstlisting}[style=yamlstyle, caption={Configuración del PCF en Docker Compose}, label=lst:docker-pcf]
  free5gc-pcf:
    container_name: pcf
    build:
      context: ./nf_pcf
      args:
        DEBUG_TOOLS: "false"
    command: ./pcf -c ./config/pcfcfg.yaml
    expose:
      - "8000"
    volumes:
      - ./config/pcfcfg.yaml:/free5gc/config/pcfcfg.yaml
      - ./cert:/free5gc/cert
    environment:
      GIN_MODE: release
    networks:
      macvlan_net:
        ipv4_address: 172.18.0.54
        mac_address: "72:b2:53:59:fc:ce"
        aliases:
          - pcf.free5gc.org
    extra_hosts:
     - "amf.free5gc.org:172.18.0.20"
     - "smf.free5gc.org:172.18.0.22"
     - "upf.free5gc.org:172.18.0.23"
    depends_on:
      - free5gc-nrf

\end{lstlisting}
\subsubsubsection{UDM}
\begin{lstlisting}[style=yamlstyle, caption={Configuración del UDM en Docker Compose}, label=lst:docker-udm]
  free5gc-udm:
    container_name: udm
    build:
      context: ./nf_udm
      args:
        DEBUG_TOOLS: "false"
    command: ./udm -c ./config/udmcfg.yaml
    expose:
      - "8000"
    volumes:
      - ./config/udmcfg.yaml:/free5gc/config/udmcfg.yaml
      - ./cert:/free5gc/cert
    environment:
      GIN_MODE: release
    networks:
      macvlan_net:
        ipv4_address: 172.18.0.55
        mac_address: "9a:a1:ed:b3:60:8c"
        aliases:
          - udm.free5gc.org
    extra_hosts:
     - "amf.free5gc.org:172.18.0.20"
     - "smf.free5gc.org:172.18.0.22"
     - "upf.free5gc.org:172.18.0.23"
    depends_on:
      - db
      - free5gc-nrf
\end{lstlisting}
\subsubsubsection{UDR}
\begin{lstlisting}[style=yamlstyle, caption={Configuración del UDR en Docker Compose}, label=lst:docker-udr]
  free5gc-udr:
    container_name: udr
    build:
      context: ./nf_udr
      args:
        DEBUG_TOOLS: "false"
    command: ./udr -c ./config/udrcfg.yaml
    expose:
      - "8000"
    volumes:
      - ./config/udrcfg.yaml:/free5gc/config/udrcfg.yaml
      - ./cert:/free5gc/cert
    environment:
      DB_URI: mongodb://db/free5gc
      GIN_MODE: release
    networks:
      macvlan_net:
        ipv4_address: 172.18.0.56
        mac_address: "ea:b1:9d:90:a4:d1"
        aliases:
          - udr.free5gc.org
    extra_hosts:
     - "amf.free5gc.org:172.18.0.20"
     - "smf.free5gc.org:172.18.0.22"
     - "upf.free5gc.org:172.18.0.23"
    depends_on:
      - db
      - free5gc-nrf
\end{lstlisting}
\subsubsubsection{Despliegue del CHF}
\begin{lstlisting}[style=yamlstyle, caption={Configuración del CHF en Docker Compose}, label=lst:docker-chf]
  free5gc-chf:
    container_name: chf
    build:
      context: ./nf_chf
      args:
        DEBUG_TOOLS: "false"
    command: ./chf -c ./config/chfcfg.yaml
    expose:
      - "8000"
      - "2122"
    volumes:
      - ./config/chfcfg.yaml:/free5gc/config/chfcfg.yaml
      - ./cert:/free5gc/cert
    environment:
      DB_URI: mongodb://db/free5gc
      GIN_MODE: release
    networks:
      macvlan_net:
        ipv4_address: 172.18.0.57
        mac_address: "ee:e6:15:22:6f:ad"
        aliases:
          - chf.free5gc.org
    extra_hosts:
     - "amf.free5gc.org:172.18.0.20"
     - "smf.free5gc.org:172.18.0.22"
     - "upf.free5gc.org:172.18.0.23"
    depends_on:
      - db
      - free5gc-nrf
      - free5gc-webui
\end{lstlisting}
\subsubsubsection{Despliegue del NEF}
\begin{lstlisting}[style=yamlstyle, caption={Configuración del NEF en Docker Compose}, label=lst:docker-nef]
  free5gc-nef:
    container_name: nef
    build:
      context: ./nf_nef
      args:
        DEBUG_TOOLS: "false"
    command: ./nef -c ./config/nefcfg.yaml
    expose:
      - "8000"
    volumes: 
      - ./config/nefcfg.yaml:/free5gc/config/nefcfg.yaml
      - ./cert:/free5gc/cert
    environment:
      GIN_MODE: release
    networks:
      macvlan_net:
        ipv4_address: 172.18.0.58
        mac_address: "e2:66:f3:0f:e0:e2"
        aliases:
          - nef.free5gc.org
    extra_hosts:
     - "amf.free5gc.org:172.18.0.20"
     - "smf.free5gc.org:172.18.0.22"
     - "upf.free5gc.org:172.18.0.23"
    depends_on:
      - db
      - free5gc-nrf
\end{lstlisting}
\subsubsubsection{Despliegue del WebUI}
A pesar de que el contenedor WebUI no es un Network Function (NF) del Core 5G SA, se ha incluido en el mismo archivo Docker Compose para facilitar su despliegue y gestión. El WebUI proporciona una interfaz gráfica para monitorizar y gestionar el Core 5G, facilitando la administración de la red.
\begin{lstlisting}[style=yamlstyle, caption={Configuración del WebUI en Docker Compose}, label=lst:docker-webui]
  free5gc-webui:
    container_name: webui
    build:
      context: ./webui
      args:
        DEBUG_TOOLS: "false"
    command: ./webui -c ./config/webuicfg.yaml
    expose:
      - "2121"
    volumes:
      - ./config/webuicfg.yaml:/free5gc/config/webuicfg.yaml
    environment:
      - GIN_MODE=release
    ports:
      - "5000:5000"
    networks:
      macvlan_net:
        ipv4_address: 172.18.0.59
        mac_address: "2a:5c:48:5c:aa:a0"
        aliases:
          - webui
    extra_hosts:
     - "amf.free5gc.org:172.18.0.20"
     - "smf.free5gc.org:172.18.0.22"
     - "upf.free5gc.org:172.18.0.23"
    depends_on:
      - db
      - free5gc-nrf
\end{lstlisting}
\subsubsubsection{Configuracion de Red y Volumenes}
\begin{lstlisting}[style=yamlstyle, caption={Configuración de red y volúmenes en Docker Compose}, label=lst:docker-network-volumes]
networks:
  macvlan_net:
   external: true
   
volumes:
  dbdata:
\end{lstlisting}


\subsubsection{Construccion y despliegue}
Una vez definido el archivo Docker Compose con la configuración de todas las NFs de Core-Host, se procede a construir las imagenes de acuerdo con el archivo Docher Compose como se puede apreciar en la figura \ref{fig:build-core-host}.
\begin{figure}[H]
    \centering
    \includegraphics[width=1.0\textwidth]{images/docker-build.png}
    \caption{Construcción de imágenes en Core-Host}
    \label{fig:build-core-host}
\end{figure}
Luego de construir las imagenes y como se puede apreciar en la figura \ref{fig:up-core-host}, se procede a desplegar los contenedores de los cuales se destaca el log del NRF, quien orquesta todo el Core 5G basado en la arquitectura SBI descrita en la figura \ref{fig:sbi_p2p}. El resto de logs de los demas contenedores no se muestran por cuestiones de espacio.
\begin{figure}[H]
    \centering
    \includegraphics[width=1.0\textwidth]{images/core-host-up.png}
    \caption{Despliegue de contenedores en Core-Host}
    \label{fig:up-core-host}
\end{figure}

%----------------------------------------------------------------
\subsection{Configuración de AMF}
El proceso de obtención, compilación y despliegue del AMF es similar al realizado para el Core-Host. Para fines de ahorrar espacio, no se mostrará el proceso de clonacion. En cambio se mostrará la construcción de la imagen Docker del AMF y su posterior despliegue.
\subsubsection{Construcción de la imagen Docker del AMF}
\begin{figure}[H]
    \centering
    \includegraphics[width=1.0\textwidth]{images/amf-docker-build.png}
    \caption{Construcción de la imagen Docker del AMF}
    \label{fig:amf-docker-build}
\end{figure}

\subsubsection{Despliegue del contenedor AMF}
Para el despliegue del contenedor AMF, se crea un archivo Docker Compose llamado \textit{docker-compose-build-amf.yaml} con la configuración del AMF mostrado en el siguiente código \ref{lst:docker-amf}.
\begin{lstlisting}[style=yamlstyle, caption={Configuración del Docker Compose AMF}, label=lst:docker-amf]
services:
  free5gc-amf:
    container_name: amf
    build:
      context: ./nf_amf
      args:
        DEBUG_TOOLS: "false"
    command: ./amf -c ./config/amfcfg.yaml
    expose:
      - "8000"
      - "38412"
    volumes:
      - ./config/amfcfg.yaml:/free5gc/config/amfcfg.yaml
      - ./cert:/free5gc/cert
    environment:
      GIN_MODE: release
    ports:
     - "38412:38412"
     - "8000:8000"
    networks:
      macvlan_net:
        ipv4_address: 172.18.0.20
        mac_address: f2:ff:cb:34:07:b5
        aliases:
          - amf.free5gc.org
    extra_hosts:
      - "nrf.free5gc.org:172.18.0.51"
      - "udr.free5gc.org:172.18.0.56"
      - "pcf.free5gc.org:172.18.0.54"
      - "ausf.free5gc.org:172.18.0.52"
      - "udm.free5gc.org:172.18.0.55"
      - "nssf.free5gc.org:172.18.0.53"
      - smf.free5gc.org:172.18.0.22"
      - "upf.free5gc.org:172.18.0.24"
      - "gnb.free5gc.org:172.18.10.3"
networks:
  macvlan_net:
   external: true
\end{lstlisting}


\subsubsection{Registro del AMF en el NRF}
Una vez desplegado el contenedor AMF, se verifica en los logs del NRF que el AMF se haya registrado correctamente en el NRF mediante la interfaz SBI de acuerdo con la arquitectura mostrada en la figura \ref{fig:sbi_p2p}. Ademas, que empiece a escuchar en el puerto SCTP 38412 y que este listo para recibir conexiones desde el NGRAN, como se muestra en la figura \ref{fig:amf-registro-nrf}.
\begin{figure}[H]
    \centering
    \includegraphics[width=1.0\textwidth]{images/amf-registro-nrf.png}
    \caption{Registro del AMF en el NRF}
    \label{fig:amf-registro-nrf}
\end{figure}

En la figura \ref{fig:amf-nrf-log} se muestran los logs del NRF que confirman que el AMF se ha registrado correctamente y que esta escuchando en el puerto SCTP 38412, listo para recibir conexiones del NGRAN.
\begin{figure}[H]
    \centering
    \includegraphics[width=1.0\textwidth]{images/amf-nrf-log.png}
    \caption{Verificación del registro del AMF en el NRF}
    \label{fig:amf-nrf-log}
\end{figure}
Las configuraciones del AMF, como las direcciones IP y puertos de escucha, se realizan en el archivo de configuración \textit{amfcfg.yaml} ubicado en el directorio \textit{./config/} del repositorio del AMF, las cuales no se muestran en este documento por cuestiones de espacio.







\subsection{Configuración de SMF}
A diferencia de los NFs desplegados en el Core-Host y SMF, tanto gNB (UERANSIM), SMF como UPF requieren de la version kernel 5.0.0-23-generic o superior a la 5.4 los cuales son compatibles con el modulo de kernel GTP para 5G.
\subsubsection{Instalacion del modulo GTP}
Como primer paso antes de instalar el modulo GTP, se instala la version de kernel requerida como se muestra en la figura \ref{fig:install-kernel}.
\begin{figure}[H]
    \centering
    \includegraphics[width=1.0\textwidth]{images/install-kernel.png}
    \caption{Instalación de kernel compatible con GTP}
    \label{fig:install-kernel}
\end{figure}
Luego de instalar el kernel compatible, se edita y actualiza el archivo \textit{/etc/default/grub} para establecer la nueva version de kernel como predeterminada al iniciar el sistema, como se muestra en la figura \ref{fig:eleccion-kernel}.
\begin{figure}[H]   
    \centering
    \includegraphics[width=1.0\textwidth]{images/eleccion-kernel.png}
    \caption{Edición del archivo y actualización de /etc/default/grub}
    \label{fig:eleccion-kernel}
\end{figure}

Una vez reiniciado el sistema con la nueva version de kernel, se procede con la instalacion del modulo GTP. Primero se clona el repositorio oficial desde GitHub como se muestra en la figura \ref{fig:clone-gtp}.
\begin{figure}[H]
    \centering
    \includegraphics[width=1.0\textwidth]{images/clone-gtp.png}
    \caption{Clonación del repositorio del módulo GTP}
    \label{fig:clone-gtp}  
\end{figure}
Luego de clonar el repositorio, se limpia, compila e instala el modulo GTP en el kernel como se muestra en la figura \ref{fig:build-gtp}.
\begin{figure}[H]
    \centering
    \includegraphics[width=1.0\textwidth]{images/build-gtp.png}
    \caption{Compilación e instalación del módulo GTP}
    \label{fig:build-gtp}
\end{figure}
Luego de instalado el modulo GTP, se carga el modulo gtp5g en el kernel y luego se listan los modulos cargados para verificar que el modulo GTP se haya cargado correctamente, como se muestra en la figura \ref{fig:load-gtp}.
\begin{figure}[H]
    \centering
    \includegraphics[width=1.0\textwidth]{images/load-gtp.png}
    \caption{Carga y verificación del módulo GTP}
    \label{fig:load-gtp}
\end{figure}
Finalmente, se revisa en los logs del sistema que el modulo GTP se haya cargado correctamente al iniciar el sistema, como se muestra en la figura \ref{fig:check-gtp}.
\begin{figure}[H]
    \centering
    \includegraphics[width=1.0\textwidth]{images/check-gtp.png}
    \caption{Verificación del módulo GTP en los logs del sistema}
    \label{fig:check-gtp}
\end{figure}

\subsubsection{Obtención e instalación del código fuente SMF}
Este proceso es similar al realizado para el Core-Host, y el AMF. Para fines de ahorrar espacio, no se mostrará el proceso de clonacion y construccion de imagen Docker del SMF.

\subsubsection{Configuración de Docker Compose}
Al igual que AMF, para el despliegue del contenedor SMF, se crea un archivo Docker Compose llamado \textit{docker-compose-build-smf.yaml} con la configuración del SMF mostrado en el siguiente código \ref{lst:docker-smf}.
\begin{lstlisting}[style=yamlstyle, caption={Configuración del Docker Compose SMF}, label=lst:docker-smf]
services:
  free5gc-smf:
    container_name: smf
    build:
      context: ./nf_smf
      args:
        DEBUG_TOOLS: "false"
    command: ./smf -c ./config/smfcfg.yaml -u ./config/uerouting.yaml
    expose:
      - "8000"
    volumes:
      - ./config/smfcfg.yaml:/free5gc/config/smfcfg.yaml
      - ./config/uerouting.yaml:/free5gc/config/uerouting.yaml
      - ./cert:/free5gc/cert
    environment:
      GIN_MODE: release
    ports:
     - "8000:8000"
    networks:
      macvlan_net:
        ipv4_address: 172.18.0.22
        mac_address: ca:76:58:9c:8e:ec
        aliases:
          - smf.free5gc.org
    extra_hosts:
      - "nrf.free5gc.org:172.18.0.51"
      - "udr.free5gc.org:172.18.0.56"
      - "pcf.free5gc.org:172.18.0.54"
      - "ausf.free5gc.org:172.18.0.52"
      - "udm.free5gc.org:172.18.0.55"
      - "nssf.free5gc.org:172.18.0.53"
      - "amf.free5gc.org:172.18.0.20"
      - "upf.free5gc.org:172.18.0.24"
      - "gnb.free5gc.org:172.18.10.3"
      
networks:
  macvlan_net:
   external: true
\end{lstlisting}

Al igual que en los NFs anteriores, se le especifica las direcciones IP de los demas NFs para que pueda resolver los nombre de dominio de cada NF mediante el archivo \textit{/etc/hosts} del contenedor SMF de acuerdo con el LLD mostrado en la tabla \ref{tab:lld_containers}.

\subsubsection{Registro del SMF en el NRF y conexión N4 con UPF}

Una vez desplegado el contenedor SMF, se verifica que se haya registrado correctamente en el NRF y que se establezca la conexion con el UPF mediante la interfaz N4, orquestado por el NRF segun la arquitectura SBI mostrada en la figura \ref{fig:sbi_p2p}.
Como se puede apreciar en la figura \ref{fig:smf-registro-nrf}, el SMF se ha registrado correctamente en el NRF y esta listo para gestionar las sesiones de usuario y la conexion N4 con el UPF esta establecida correctamente.
\begin{figure}[H]
    \centering
    \includegraphics[width=1.0\textwidth]{images/smf-registro-nrf.png}
    \caption{Registro del SMF en el NRF}
    \label{fig:smf-registro-nrf}
\end{figure}

Logs logs del establecimiento de la conexion N4 entre el SMF y el UPF se muestran a detalle en la session del UPF en la figura \ref{fig:smf-upf-n4-connection}.






\subsection{Configuración de UPF}
El proceso de obtención, compilación, instalación de modulo gtp5g y despliegue del UPF es similar al realizado para el Core-Host, AMF y SMF. Para fines de ahorrar espacio, no se mostrará el proceso de clonacion y construcción de imagen Docker del UPF.
\subsubsection{Configuración de Docker Compose}
Al igual que AMF y SMF, para el despliegue del contenedor UPF, se crea un archivo Docker Compose llamado \textit{docker-compose-build-upf.yaml} con la configuración del UPF mostrado en el siguiente código \ref{lst:docker-upf}.
\begin{lstlisting}[style=yamlstyle, caption={Configuración del Docker Compose UPF}, label=lst:docker-upf]
version: "3.8"
services:
  free5gc-upf:
    container_name: upf
    build:
      context: ./nf_upf
      args:
        DEBUG_TOOLS: "false"
    command: bash -c "./upf-iptables.sh && ./upf -c ./config/upfcfg.yaml"
    expose:
      - "8000"
      - "38412"
    volumes:
      - ./config/upfcfg.yaml:/free5gc/config/upfcfg.yaml
      - ./config/upf-iptables.sh:/free5gc/upf-iptables.sh
    cap_add:
      - NET_ADMIN
    ports:
     - "38412:38412"
     - "8000:8000"
    networks:
      macvlan_net:
        ipv4_address: 172.18.0.24
        mac_address: be:44:98:a4:e5:c4
        aliases:
          - upf.free5gc.org
    extra_hosts:
      - "nrf.free5gc.org:172.18.0.51"
      - "udr.free5gc.org:172.18.0.56"
      - "pcf.free5gc.org:172.18.0.54"
      - "ausf.free5gc.org:172.18.0.52"
      - "udm.free5gc.org:172.18.0.55"
      - "nssf.free5gc.org:172.18.0.53"
      - "amf.free5gc.org:172.18.0.20"
      - "smf.free5gc.org:172.18.0.22"
      - "gnb.free5gc.org:172.18.10.3"
networks:
  macvlan_net:
   external: true

\end{lstlisting}
\subsubsection{Registro del UPF en el NRF}
Una vez desplegado el contenedor UPF, en la figura \ref{fig:upf-up} se aprecia que prepara los parametros para la interfaz N3 con gNB y los recursos para el UE, los cuales son \textit{10.60.0.0/16} y \textit{10.61.0.0/16}.

\begin{figure}[H]
    \centering
    \includegraphics[width=1.0\textwidth]{images/upf-up.png}
    \caption{ UPF registro y preparacion de interfaz N3}
    \label{fig:supf-up}    
\end{figure}

\subsubsection{ Conección N4 con SMF}
En la figura \ref{fig:smf-upf-n4-connection} se aprecia que el UPF establece la conexion N4 (PFCP) con el SMF y queda listo para gestionar las sesiones de usuario y el trafico de datos del UE.
%figura 5.19
\begin{figure}[H]
    \centering
    \includegraphics[width=1.0\textwidth]{images/smf-upf-n4-connection.png}
    \caption{ UPF registro y preparacion de interfaz N3}
    \label{fig:smf-upf-n4-connection}    
\end{figure}












\subsection{Integración con UERANSIM}

\section{Implementación de la red de distribución P4}
\subsection{Diseño de los programas P4}
\subsection{Tablas y acciones configuradas}
\subsection{Inserción INT-MD en tráfico N2 (SCTP)}
\subsection{Inserción INT-MD en tráfico N3 (GTP-U)}

\section{Servidor de recolección y análisis}
\subsection{Implementación del sink INT}
\subsection{Procesamiento y parsing de metadatos}
\subsection{Modelo de base de datos}
\subsection{Visualización en Grafana}

\section{Tecnologías empleadas}
\subsection{Docker / Docker Compose}
\subsection{GNS3 / GNS3 VM}
\subsection{Wireshark y herramientas auxiliares}



\section{Despliegue del NGRAN con UERANSIM}
El NGRAN se ha desplegado utilizando la herramienta UERANSIM, la cual simula tanto el gNB como el UE en modo 5G Standalone (SA). UERANSIM se ha configurado para conectarse al Core 5G SA desplegado previamente, estableciendo el enlace N2 (SCTP) con el AMF y el enlace N3 (GTP-U) con el UPF.
\subsection{Despliegue de UERANSIM}
El despliegue de UERANSIM se realiza mediante Docker Compose, creando un archivo llamado \textit{docker-compose-build-ngran.yaml} que contiene la configuración tanto del gNB como del UE. El archivo Docker Compose se muestra en el siguiente código \ref{lst:docker-ueransim}.
\begin{lstlisting}[style=yamlstyle, caption={Configuración del Docker Compose UERANSIM}, label=lst:docker-ueransim]
version: "3.8"
services:
  ueransim:
    container_name: ueransim
    build:
      context: ./ueransim
    command: ./nr-gnb -c ./config/gnbcfg.yaml
    expose:
      - "38412"
    volumes:
      - ./config/gnbcfg.yaml:/ueransim/config/gnbcfg.yaml
      - ./config/uecfg.yaml:/ueransim/config/uecfg.yaml
    cap_add:
      - NET_ADMIN
    devices:
      - "/dev/net/tun"
    ports:
     - "38412:38412"
    networks:
      macvlan_net:
        ipv4_address: 172.18.10.3
        mac_address: f6:b8:eb:4d:0f:3c
        aliases:
          - gnb.free5gc.org
    extra_hosts:
      - "amf.free5gc.org:172.18.0.20"
      - "upf.free5gc.org:172.18.0.24"   
networks:
  macvlan_net:
   external: true
\end{lstlisting}

\subsection{Configuración del gNB}
La configuración del gNB se realiza mediante el archivo \textit{gnb.yaml}, donde se especifican los parámetros necesarios para la conexión con el Core 5G SA, como la dirección IP del AMF, el PLMN ID, y los parámetros de la interfaz N3. Un ejemplo de configuración del gNB se muestra en la figura \ref{fig:gnb-config}.
\begin{figure}[H]
    \centering
    \includegraphics[width=1.0\textwidth]{images/gnb-config.png}
    \caption{Configuración del gNB en UERANSIM}
    \label{fig:gnb-config}
\end{figure}
\subsection{Configuración del UE}
La configuración del UE se realiza mediante el archivo \textit{ue.yaml}, donde se especifican los parámetros necesarios para la conexión con el gNB, como el IMSI, la clave de seguridad, y los parámetros de la interfaz N1. Un ejemplo de configuración del UE se muestra en la figura \ref{fig:ue-config}.
\begin{figure}[H]
    \centering
    \includegraphics[width=1.0\textwidth]{images/ue-config.png}
    \caption{Configuración del UE en UERANSIM}
    \label{fig:ue-config} 
\end{figure}

\chapter{Resultados Experimentales}

%\section{Metodología de evaluación}
%\subsection{Escenarios de prueba}
%\subsection{Tráfico analizado}
%\subsection{Métricas recolectadas}

\section{Validación funcional}
\subsection{INT en tráfico N2}
\subsection{INT en tráfico N3}
\subsection{Recepción de metadatos en el servidor}

\section{Resultados cuantitativos}
\subsection{Latencia hop-by-hop}
\subsection{Carga adicional introducida por INT}
\subsection{Impacto en el plano de usuario}

\section{Resultados cualitativos}
\subsection{Dashboards obtenidos}
\subsection{Interpretación de tendencias}
\subsection{Problemas encontrados}

\chapter{Discusión}

\section{Análisis crítico de los resultados}
\subsection{Comparación con el estado del arte}
\subsection{Validación de la hipótesis planteada}

\section{Beneficios del uso de P4 en redes 5G}
\section{Desafíos de escalabilidad}
\section{Consideraciones de seguridad para INT}
\section{Limitaciones del trabajo realizado}

\chapter{Conclusiones y Trabajo Futuro}

\section{Conclusiones principales}
\section{Contribuciones del trabajo}
\section{Limitaciones del estudio}
\section{Líneas futuras de investigación}
\subsection{INT en 6G}
\subsection{UPF programable con P4}
\subsection{IA para análisis de telemetría}



% Apéndices
\appendix
\chapter{Apéndices}
\section{Código}
\section{Topologías de red}
\section{Configuraciones del core 5G}
\section{Capturas de tráfico}
\section{Dashboards de Grafana}
\section{Scripts y herramientas auxiliares}

% Bibliografía
\bibliographystyle{plain}
\bibliography{chapters/references}


\end{document}
