\chapter{Telemetría y QoS en 5G}
En 5GC, la telemetria juega un papel crucial en la monitorización y gestión del rendimiento de la red, especialmente en el contexto de las demandas crecientes de servicios como URLLC (Ultra-Reliable Low-Latency Communications), eMBB (enhanced Mobile Broadband) y mMTC (massive Machine Type Communications). La telemetría en banda (In-Band Network Telemetry, INT) permite la recopilación de datos en tiempo real directamente desde los paquetes que atraviesan la red, proporcionando una visibilidad detallada del estado de la red y facilitando la detección y resolución de problemas.

\section{Desafíos de Telemetría en Redes 5G}
\subsection{Limitaciones de mecanismos tradicionales (SNMP, NetFlow, sFlow)}
\subsection{Requisitos de visibilidad en URLLC, eMBB y mMTC}

\section{In-Band Network Telemetry (INT)}
\subsection{Concepto y motivación}
\subsection{Arquitectura INT: source, transit y sink}
\subsection{Metadatos INT-MD y su estructura}
\subsection{INT vs telemetría out-of-band}
%\section{Telemetría aplicada en 5G}
%\subsection{Relevancia de INT en el plano de control N2}
%\subsection{Relevancia de INT en el plano de usuario N3}
%\subsection{Métricas clave para tráfico GTP-U}
%\subsection{Desafíos y limitaciones en redes móviles}
%--- QoS en 5G ---
\section{Calidad de Servicio en 5G}
\section{QoS basado en 5G Flujo}
\subsection{Definición de flujo en 5G}
\subsection{Identificación y clasificación de flujos}
\subsection{Mecanismos de gestión de QoS por flujo}
\section{Señalización de QoS en 5G}
\section{Caracteristicas de QoS en 5G}